% - submitted to ... .
% Define metric tensor
% This is patterned after llncs.tex, the documentation file of
% the LaTeX2e class from Springer-Verlag
% for Lecture Notes in Computer Science, version 2.4
\documentclass{llncs}
\usepackage{llncsdoc}
\usepackage{sfmath}

% Dirac bra-ket symbols
\def\bra#1{\mathinner{\langle{#1}|}}
\def\ket#1{\mathinner{|{#1}\rangle}}
\newcommand{\braket}[2]{\langle #1|#2\rangle}  % \braket{\psi}{\Psi}
\def\Bra#1{\left<#1\right|}
\def\Ket#1{\left|#1\right>}

%\newcites
%\cite
\newcommand{\rmssf}[1]{\relax\ifmmode\mathsf{#1}\else\textsf{#1}\fi}

\begin{document}

\title{Semantic Elements of Physics Markup}
\author{Joseph B. Collins}
\institute{Naval Research Laboratory\\ 4555 Overlook Ave, SW\\ Washington, DC  20375-5337}

\maketitle

\begin{abstract}

\end{abstract}

\section{Introduction}


\section{The Elements}

\subsection{Physical Quantity}
Quantities are treated as the fundamental construct within the
Syst\`{e}me Internationale (SI), where quantity is defined as a
``property of a phenomenon, body, or substance, where the property has
a magnitude that can be expressed by means of a number and a
reference''.
From the perspective of markup, a physical quantity must first have a
name attribute, such as {\em wavelength}.
From the SI definition, one can see that the physical quantity is a
complex type, and must have a combination of attributes and contained,
nested elements, which include a {\em dimension}, a {\em kind}, and a
{\em magnitude}.
These are discussed below.

\subsubsection{Physical Dimension}
The most primitive concept of physical markup is that of {\em physical
  dimension}.
We here define physical dimension the common usage to include that
which is defined as the {\em quantity dimension} in the SI standard.
The SI nomenclature regarding quantities can sometimes be confusing
as the names of quantity dimensions are often the same as or similar
to the names of quantities.
The SI refers to the usage of the term quantity in a generic, or
abstract sense, in which case, for example, length refers not to any
specific length of any particular thing, but the general concept of
length.
The SI also refers to the usage of the term quantity in a specific
sense, where, for example, the length of a standard meter refers to
the spatial extent of a specific physical object.
Similarly, the time dimension refers to the space in which temporal
quantities, such as ``transit time'', ``period'', or ``duration'', are
represented, measured and expressed.

Examples of generic quantities are found in the seven SI Base
Quantities: length, mass, time, electric current, thermodynamic
temperature, amount of substance, and luminous intensity.
In the SI nomenclatre, for a system of quantities, the base quantities
define the quantity dimensions for that system.
Other quantities, called {\em derived quantities}, are not referred to
as dimensions in SI.
In our usage, the term physical dimension, such as used in dimensional
analysis, refers not only to the SI quantity dimension, but also to SI
generic quantities, including generic derived quantities.

We make this distinction in order to support the expression of
different systems of quantities. 
Different systems of quantities may have different base quantities,
hence different quantity dimensions, or may share some of the same
base quantities.
Essentially, each system of quantities defines a basis with which to
span either a subspace or the whole of the space of physical
dimensions.

Physical dimensions also have mathematical properties: any given
system of base quantities, the quantity dimensions, with the formal
inclusion of a dimensionless identity element, $1$, act as generators
of a commutative group under multiplication.
In addition to multiplication, each quantity dimension may be raised
to any rational power.
The base quantities of a system of quantities are formally treated as
incommensurate with respect to each other, no one base quantity being
expressible in terms of the others.

While our discussion of physical dimension to this point has been as
an attribute of a physical quantity, we also want to define physical
dimensions as xml elements.
The fact that we define physical dimensions also as elements does not
preclude us from also using the same element name values as attributes
of physical quantities.
Dimensional analysis often reduces relations between physical
quantities to relations between dimensions, so physical dimensions,
which are usually treated as attributes of physical quantities, may
also stand on their own in physical relations. 
As elements they have the following attributes: a name; the quantity
system to which it belongs; and a symbol.
We assume by default that the quantity system is the SI, but allow for
alternative systems.

{\em Derived quantities}, as defined by SI, are quantities, in a
system of quantities, which are defined in terms of the base
quantities of that system.
The manner in which derived quantities are defined is by application
of the above described mathematical operations to the base quantities.
Of the infinite number of possible derived quantities, only a finite
few have their own names, for example, area, energy, and momentum.
Colloquially, the generic sense of these quantities may be thought of as
quantity dimensions themselves, e.g., ``the kinetic energy of the proton
has the dimension of energy''.

\subsubsection{Kind of Quantity}
The next attribute of a physical quantity is that of the {\em kind}
of physical quantity.
The kind of a quantity is used to distinguish between different
quantities which have the same quantity dimensions.
The SI concedes that the concept of the kind of a quantity is to some
extent arbitrary.
Perhaps the best illustration is by way of examples.
The salinity of a solution is typically stated as a mass fraction,
i.e. the mass of dissolved salt per unit of mass of solution.
As such, salinity is a dimensionless quantity.
Angle is also dimensionless, given, for example, by the ratio of the
subtended circular arc and the radius of the same circle.
While dimensionally equivalent, one still considers these kinds of
quantities to be distinct.
There are many dimensionless quantities distinguished by kind.
Similarly the quantities moment of force and energy have the same quantity
dimension but are distinguished from each other.

\subsubsection{Magnitude of a Quantity}
The magnitude of a specific scalar physical quantity represents the
amount of that quantity, in terms of real numbers or some
subset of the real numbers, and in terms of reference quantities, or
units.
As such, the magnitude of a scalar physical quantity is the
mathematical product of a real number and a unit. 
For example, {\em the} kilogram is the reference mass that resides at
the International Bureau of Weights and Measures (BIPM) in S��vres,
France.
In the SI system, the masses of all other physical objects are
measured in proportion to that kilogram standard, where the proportion
is expressed as a limited precision real number.
There is also an accompanying error value for the real number.
By default, when not specified, the error in a number is assumed to be
half of the place value of the least significant digit expressed.

In some cases vector quantities, including complex numbers and
tensors, are used.
When vector quantities are used, they may either be heterogeneous,
where each component is a quantity, or homogeneous, where the
components are pure numbers and the unit may be represented as a
distinct factor.
The representation of error with vector quantities is
typically in terms of a covariance structure.
Finally, unambiguous representation of a limited precision real number
requires the use of an scientific notation with significand and
exponent.
It is needed to express, for example, the two significant digits of  $6.2 \times
10^{3}$, instead of writing $6,200$, the latter being ambiguous.

The current upper limit of precision of measurement for physical
constants is about twelve to thirteen significant decimal digits, such
as for the Rydberg constant.
Magnitudes of physical constants and their uncertainties range from 
$10^{-72}$ to $10^{50}$ in common physical units.
The size of the known universe is on the order of $10^{62}$ planck
lengths and the age of the universe is on the order of $10^{62}$
planck times and of the order $10^{62}$ planck masses and $10^{80}$ protons.
While it may be ideal to specify a format for arbitrary precision and
arbitrary magnitude measurable values, current IEEE double precision
format provides about $17$ significant digits and magnitudes of
$10^{308}$ and $10^{-307}$.
While it is not inconceivable that physics-based computations may
exceed the expressiveness the IEEE double precision format, for most
practical purposes, that format appears adequate 
http://www.w3.org/TR/xmlschema-2/#double [IEEE 754-1985]
or MathML <cn type=''double''>.
 
Depending on how the significand of a double precision format is
interpreted, the semantics of the precision limit of a measurement may
be lost in the conversion to and from machine double precision.
From the perspective of supporting large precision, $16$ significant
digits is a good thing.
Most measurements, however, fall far short of that level of
precision.
From a semantic perspective, the machine precision is merely the
maximum precision that may represented in a machine word: the IEEE
format does not in itself support the expression of uncertainty.
Considering that in the conversion of a double precision literal to a
machine double the limit of precision expressed in the literal is
usually lost, it seems necessary to provide a separate mechanism for
explicitly expressing the precision.

Symbol for quantity is a <m:ci>

\section{Summary}


\begin{thebibliography}{1}



\end{thebibliography}

\end{document}



