\documentclass{article}
\usepackage[show]{macros/ed}
\usepackage[savemem]{macros/listings/listings}
\usepackage{lstpatch,lstomdoc}
\usepackage[hide]{xmlindex}
\usepackage[fancyhdr,today,draft]{macros/svninfo}
\pagestyle{fancyplain}\fancyhead[RE,LO]{\leftmark}\fancyhead[LE,RO]{\thepage}

\def\chemml{{\sc{CML}}}
\def\physml{{\sc{PhysML}}}
\def\ocdoc{{\sc{OChemDoc}}}
\def\om#1{\fbox{\ensuremath{#1}}}
\lstset{float=htb,columns=flexible,frame=lines,language=[omdoc]XML,basicstyle=\scriptsize,
  indexstyle=\indextt,indexstyle=[1]\indexelement,indexstyle=[2]\indexattribute,showstringspaces=false}

 \title{Integrating CML and OMDoc in {\ocdoc}}
\author{Michael Kohlhase\\
  Jacobs University
\and 
Peter Murray Rust\\
University OF Cambridge }
\begin{document}
\svnInfo $Id: ochemdoc.tex 19 2007-11-30 08:12:44Z pmr $
\svnKeyword $HeadURL:https://svn.omdoc.org/repos/omdoc/branches/omdoc-1.2/projects/omdoc-2.0/OpenMath-paper/presel.tex $
\maketitle
\begin{abstract}
  This will revolutionize science.\ednote{continue}
\end{abstract}

\section{Introduction}

The {\openmath}~\cite{BusCapCar:2oms04} and {\mathml}~\cite{CarIon:MathML03} have
introduced the notion of a ``content dictionary'' to give meaning to symbols in
mathematical formulae. Content dictionaries are documents that encode the mathematical
knowledge necesssary to understand the meaning of a symbol. Thus content dictionaries make
the context of mathematical formulae explicit and allow the structuring of
mathematids. The {\omdoc} format~\cite{Kohlhase:omdoc1.2} extends the rather simple native
content dictionary representation format of {\openmath} with an infrastructure for
structuring, inheriting and documenting content dictionaries.

The {\omdoc} format has been extended to a markup language for physics by adding
observables, systems, and experiments. We will build on this and add chemical molecules,
etc. to this mix from {\chemml}~\cite{CML:web} to create a markup language {\ocdoc} for
knowledge in chemistry.

Our design goals in this endeavor are
\begin{enumerate}
\item {\ocdoc} must allow a transparent transition from chemical colloquialism to physical
  precision and on to mathematical rigor in representation.
\item {\ocdoc} must represent objects and context compositionally.
\item {\ocdoc} must be human readable/authorable in the small.
\item {\ocdoc} must be referentially transparent.
\item {\ocdoc} must distinguish between objects and their encodings. 
\end{enumerate}


\section{Examples}
We consider an example from Steininger's Dissertation~\cite[p.23]{Steininger:pke05}
\lstinputlisting{../examples/diss215.ocdoc}

\section{Conclusion}
\bibliographystyle{alpha}
\bibliography{bibs/kwarc,ochemdoc}
\newpage
\begin{appendix}
  \lstinputlisting[language=RNC,nolol,frame=none]{../examples/ochemdoc.rnc}
\end{appendix}
\end{document}

%%% Local Variables: 
%%% mode: stex
%%% TeX-master: t
%%% End: 
