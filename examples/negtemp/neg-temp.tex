\documentclass[12pt]{article}
\usepackage{stex,physml}
\usepackage{color}
\usepackage[latin1]{inputenc}
\usepackage{epsfig}
\usepackage[german]{babel}
\usepackage{amsmath}
\textheight23cm
\begin{document}
%%%%%%%%%%%%%%%%%%%%%%%%%
% Silbentrennung
\hyphenation{
        mak-ro-sko-pischen
}
%\setcounter{errorcontextlines}{\maxdimen}
 \parindent 0pt
\title{Seminar: Statistical Physics\\``Negative Temperatures''}
\author{J"orn Otten\and Heinrich Stamerjohanns\and Jan Kickstein}
\date{12. Februar 1995}
\maketitle
%\tableofcontents 
%\newpage
\section{Introduction}
We will talk about the phenomenon of \textsl{Negative Temperatures}.
At first, we will repeat the statistical notion of temperature. Then
we will explain the ideal paramagnetic crystal, whose temperature may
become negative, as the most simple system, that can be applied by statistical
mechanics as an introductory example.
Finally we will look at the theoretical impact of negative temperatures,
as well at the experimental conditions, to obtain these.

\section{Statististical Definition of Temperature}

\subsection{Thermal Interaction between two 
            Macroscopic Systems $A_1$ and $A_2$}

\requiremodules[exclude]{background}
\begin{module}[id=mytemparature,uses=probability-intro]
\symdef{sumofstates}[2]{\Omega_{#1}(#2)}

We define the temperature of the thermal interaction between two 
macroscopic systems $A_1$ und $A_2$.

May the energies of the system be $E_1$ und $E_2$. 

The energy-scale is divided into intervals of $\delta E_1$ and $\delta
E_2$ respectively.

\begin{definition}[display=flow]
  $\sumofstates{1}{E}$ is the number of states in the range $E_1+\delta E_1$, similarly for
  $A_2$.
\end{definition}

Both systems are not thermally isolated, i.e. heat transfer to exchange
energy is possible (it is assumed that the external parameters of the system 
are fixed invariable, therefore the energy exchange occurs as heat transfer)
But the compound system $A=A_1+A_2$ is isolated, i.e. $E=E_1+E_2$ is constant.
It it assumed that both systems are in equilibrium.
The energies of $A_1$ may take a value in a big range,
but these do not appear with the same probability.
If $A_1$ carries energy $E_1$, then $A_2$ has energy $E_2=E-E_1$.
The number of accessible states to system A can be seen as a function of $E_1$
i.e. $\sumofstates{}{E_1}$ is the number of states of A, if $A_1$ has energy $E_1$.
 The fundamental postulate requires: In equilibrium A is equally probable
in each of its states. The probability $\probability{E_1}$, that $A_1$ 
has the energy $E_1$ proportional to $\sumofstates{}{E_1}$.
\begin{equation}
\probability{}{E_1} = \constant \cdot \sumofstates{}{E_1}=\frac{\sumofstates{}{E_1}}{\sumofstates{tot.}{}}
\end{equation}
Here $\sumofstates{tot.}{}$ is the total number of the accessible states to A.
\begin{equation}
\sum{tot.}=\sum_{E_1}\sumofstates{}{E_1}
\end{equation}
If $A_1$ has the energy $E_1$,
then this system may reside in any of its states
$\sumofstates{1}{E_1}$. Simultaneously $A_2$ has to carry the  
energy $E-E_1$, and is in one of its $\sumofstates{2}{E-E_1}$ states.

Since each of the states $A_1$ can be combined with each of $A_2$ 
to form a states A, is the number of accessible states to A, 
if $A_1$ has the $E_1$ 

\begin{equation}
\sumofstates{}{E_1}=\sumofstates{1}{E_1} \cdot \sumofstates{2}{E-E_1}\Rightarrow P(E_1)=C \cdot \sumofstates{1}{E_1}
  \cdot \sumofstates{2}{E-E_1}
\end{equation}

Now we analyze the dependability of $P(E_1)$ of $E_1$.
If $E_1$ and $E_2$ have very many degrees of freedom, then 
$\sumofstates{1}{E_1}$ and $\sumofstates{2}{E_2}$ are rapid growing functions 
(for large f: $\sumofstates{}{E}\sim E^f$).
If $E_1$ grows, then $\sumofstates{1}{E_1}$ grows rapidly, 
while $\sumofstates{2}{E-E_1}$ rapidly decreases. 
$P(E_1)$ there has a strong maximum for a $E_1=\bar{E}$. 
This maximum we would like to determine.

For this we determine the maximum of $lnP(E_1)$:
\begin{equation}
 \frac{\partial lnP}{\partial E_1}=\frac{1}{P} \cdot \frac{\partial P}{\partial E_1}=0
\end{equation}
With\\
\begin{equation}
 lnP(E_1)=lnC+ln\sumofstates{1}{E_1}+\ln\sumofstates{2}{E_2}
\end{equation}
the result is\\
\begin{equation}
 \frac{\partial \ln\sumofstates{1}{E_1}}{\partial E_1}+\frac{\partial \ln\sumofstates{2}{E_2}}
{\partial E_2} \cdot (-1)=0 \mbox{ mit }dE_1=-1 \cdot dE_2
\end{equation}
or\\
\begin{equation}
 \beta_1(\bar{E_1})=\beta_2(\bar{E_2})
\end{equation}
Here $\bar{E_1}$ and $\bar{E_2}$ are energies of $A_1$ and $A_2$ 
at the maximum and 
\begin{equation}
\beta (\bar{E_{1,2}})=\frac{\partial \ln\sumofstates{}{}}{\partial E}
\end{equation} with  
the dimension reziprocal energy.\\

Now we define the dimensionless parameter \textbf{temperature T}:
\begin{equation}
kT:=\frac{1}{\beta}
\end{equation} \\
Here k is a positive constant with the dimension energy.\\
Further on the \textbf{Entropie S} is defined by:
\begin{equation}
S=k \cdot \ln\sumofstates{}{}
\end{equation}
From this it follows:
\begin{equation}
\frac{1}{T}=\frac{\partial S}{\partial E}
\end{equation}
The conditions for a maximum can be rewritten as:
\begin{equation}
S_1+S_2=Max \mbox{ oder } T_1=T_2
\end{equation}
Therefore we derive two basic principles:
\begin{itemize}
\item If two systems are in equilibrium apart, and $\beta$ has the same
value in both systems, they will stay in equilibrium on thermal contact
\item If $\beta$ has a different value in both system, they will not
stay in equilibrium if brough into thermal contact.
\end{itemize}

Regarding three systems A, B, C, which are apart in equilibrium.
Then:
\begin{equation}
\beta_A=\beta_C \mbox{, } \beta_B=\beta_C\, \Rightarrow \beta_A=\beta_B
\end{equation}

This is also known as the \textbf{Zeroeth Theorem}:\\[0.5cm]
If two equilibrium systems are in equlibrium with a third, then
they are also with each other.\\[0.5cm]

Therefore we can deduce the possibility of test systems (thermometer).\\
This should suffice as a repetition of the statistical definition of temperature.
\footnote{see F. Reif: Statistische Physik und Theorie der 
W"arme}.
\\
Recapulating the \textbf{Absolute Temperature T} is given by
\begin{equation}
\frac{1}{kT}=\beta=\frac{\partial \ln\sumofstates{}{}}{\partial E}
\end{equation}
.\\
Since in general $\sumofstates{}{E}$ rapidliy growing function of E, 
it is true that: 
\begin{equation}
\beta>1 \mbox{ bzw. } T>O
\end{equation}
But that this is not always true, will be now shown.

\section{Analysis of an ideal paramagnetic crystal in an external magnetic field}
\subsection{Model for the ideal paramagnetic crystal with spin 1/2}

Since the definition of a spin temperature for a systems that allow more than
two spin states is quite elaborate, we restrict our description of
a system with spin $\frac{1}{2}$. We analyze a paramagnetic crystal
inside a magnetic field $\vec B$. We differentiate between the 
lattice system and the spin system. 
Because of the magnetic field there is a magnetic moment $\mu_{i}$ 
at each point of the lattice.

\begin{equation}
\mu_{i}=\gamma \vec S
d\end{equation}
$\gamma $ is the ''gyromagnetic ratio'', which describe the 
proportion between spin and magnetic moment.
The macroscopic stabilization of the spin system is accomplished 
by the spin-spin-coupling, which is negligible with regards to
the coupling of the magnetic moment and the applied field.

Such a system can be describe by the Hamiltion operator:
\begin{equation}
H=-\sum_{i=1}^{N} \vec \mu_{i} \cdot \vec B_{i}=-B \sum_{i=1}^{N} \mu_{iz}
\end{equation}

Since we neglect all other degrees of freedom, the system is fully determined,
if the spin state is known at each point of the lattice. In this case
only two states are possible per lattice point., which we denote
by $|+\rangle $ und $ |-\rangle $.
The Eigenstates of the system have a well-defined energy:
 
\begin{eqnarray}
\epsilon_{+}= - \frac{\gamma}{2}B=-\mu B
\epsilon_{-}= - \frac{\gamma}{2}B=+\mu B
\end{eqnarray}

The absolute frequencies of the individual states are described by $n_{+}$,$n_{-}$
the the total energy of the system can be described by:
\begin{equation}
E=(n_{-}+n{+})\mu B
\end{equation}

Let $N$ be the total number of lattice points, the the range of values
of the energy lies between $-N\mu B$ snd $+N\mu B$. 
If we denote this energy interval by $E$, the the absolute frequencies
are fully determined by $N$ and $E$.
Then
\begin{eqnarray}
n_{+}+n_{-}=N\\
(n_{-}-n_{+})\mu B =E \\ 
\Rightarrow
n_{+}=\frac{1}{2}(N-\frac{E}{\mu B})\\
n_{-}=\frac{1}{2}(N+\frac{E}{\mu B})
\end{eqnarray}
applies.

Of course the equations above apply only, if the system is isolated.
This, as often in our nature, a not existing ideal case, because in 
reality states of the energetically higher $|-\rangle$ can turn into
$|+\rangle$ state and send off a photon with the 
momentum  $\frac{1}{\hbar}\mu B$. But the probability of such an event
is relatively low, since the grequency of the photon lie within the 
radio frequencies and transitions in this range have a very low
probability.

If we neglect this effect of \textsl{spontanous emission}, the we can regard 
the energy as constant. As a sum, the absolute frequencies are constant in this 
approximation. We can see from eq. \ref{spintemp}, that we can easily 
define the temperature of the spin-dystem. 

\subsection{Calculation of the entropy}
The entropie of a system, whose energy E is determined, is therefore:
\begin{equation}
S=k \cdot \ln\sumofstates{}{}
\end{equation}
Hierbei ist $\sumofstates{}{}$ die Anzahl der Zust"ande, die bei gegebener Teilchenzahl 
N und einem gegebenen Magnetfeld $\vec{B}$ die Energie E besitzen. Alle 
Zust"ande gleicher Energie E haben die gleiche Anzahl $n_-$ von Spins in dem
Zustand $|-\rangle$; sie unterscheiden sich untereinander nur durch die
Lage der Spins $|-\rangle$  
in dem Kristall. Die Anzahl dieser Zust"ande ist daher
\begin{equation}
\sumofstates{}{}=C^{n_-}_N=\frac{N!}{n_-!n_+!}.
\end{equation}
Daraus ergibt sich f"ur die Entropie
\begin{equation}
S=k \cdot \ln\frac{N!}{n_-!n_+!}=k \cdot \ln\frac{N!}{[\frac{N+E/\mu B}{2}]![\frac{N-E/\mu B}
  {2}]!}.
\end{equation}
Au"ser in den begrenzten F"allen, wo eine der Zahlen $n_+$ und $n_-$ in der N"ahe 
von 1 liegen, kann man die Logarithmen der Faktoren in der Stirling-N"aherung 
schreiben:
\begin{equation}
lnN!\simeq N \cdot\ln N-N \mbox{ bzw. }\simeq n \cdot \ln N
\end{equation}
Man erh"alt damit
\begin{equation}
S=k \cdot (N \cdot \ln N-\frac{N+E/\mu B}{2} \cdot \ln\frac{N+E/\mu B}{2}-\frac{N-E/\mu B}{2}\cdot
  \ln\frac{N-E/\mu B}{2})
\end{equation}
F"ur einen gegebenen Kristall (bestimmtes N) und ein konstantes Magnetfeld 
$\vec{B}$ ist die Entropie eine Funktion der Energie.
Wenn die Energie den Minimalwert $-N\mu B$ annimmt, befinden sich alle Spins 
in dem Zustand $|+\rangle$; das System hat dann nur einen zug"anglichen Zustand und 
seine Entropie ist Null. Die gleiche "Uberlegung ist g"utig, wenn die Energie 
den Maximalwert $+N\mu B$ annimmt. Die Entropie ist daher eine zun"achst wachsende 
und dann fallende Funktion der Energie. Die Kurve ist symmetrisch, ihr Maximum 
wird f"ur E=0 erreicht und betr"agt 
\begin{equation}
S_m=N\cdot k\cdot\ln 2.
\end{equation}

\subsection{Berechnung der Temperatur}

Wie bereits gezeigt, gilt f"ur die mikrokanonische Temperatur $T^*$ 
\begin{equation}
\frac{1}{T^*} =\frac{\partial S}{\partial E}.
\end{equation}
F"ur den idealen Paramagneten erh"alt man also
\begin{equation}
\frac{1}{T^*} = \frac{k}{2 \mu B} \ln \frac{N - E/\mu B}{N+E/\mu B} =
\frac{k}{2\mu B} \ln \frac{n_+}{n_-}.
\end{equation} 
\begin{figure} 
\centerline{\epsfig{file=figures/entropie.eps,height=3in}}
\caption{Die Entropie S in einem System von $N$ unabh"angigen Spins $\frac{1}{2}$
in einem Magnetfeld $\vec B$.}
\label{entr}
\end{figure}
Sind $\mu$, $N$, $B$ konstant, so ist die Temperatur nur noch von E abh"angig.
$T^* = T(E)$. Wird E positiv, dann ist das Argument des Logarithmus $<1$. Die
Temperatur $T^*$ wird negativ. Abb. \ref{entr} zeigt den Zusammenhang deutlich.
Ab $E>0$ f"allt die Entropiekurve ab. $\frac{\partial S}{\partial E}$ ist dann
negativ.
\begin{figure} 
\centerline{\epsfig{file=figures/t_von_e.eps,height=3in}}
\caption{T als Funktion von E in einem System von $N$ unabh"angigen Spins
$\frac{1}{2}$}
\label{t}
\end{figure}
In Abb. \ref{t} ist $T^*$ als Funktion von E aufgetragen. Bei E = 0 hat die 
Funktion eine senkrechte Asymptote. 
\begin{table} 
\begin{center}
%\begin{tabular}{|r|r|r|r|r|}
\begin{tabular}{|c|c|c|c|c|}
     \hline
                  &                  &           &           &\\
     E & Zustand & $n_+ /  n_-$ & $\ln n_+ / n_-$ & T \\
                  &                  &           &           &\\
     \hline     
     \hline
                  &                  &           &           &\\
     $E_{min}    $&alle   $|+\rangle$&$\infty$   &$\infty$   &$0_+$\\
                  &keine   $|-\rangle$&          &           &     \\
     \hline     
                  &                  &           &           &\\
     $E_{min}<E<0$&viele  $|+\rangle$&$>1$       &$>0$       &$\to\infty$ \\
     $(steigend) $&wenige $|-\rangle$&$(fallend)$&$(fallend)$&$(steigend)$\\ 
     \hline
                  &                  &           &           &\\
     $0$          &$50\%\;|+\rangle$ &$1$        &$0$        &$Sprungstelle$\\
                  &$50\%\;|-\rangle$ &           &           &\\
     \hline 
                  &                  &           &           &\\
     $0<E<E_{max}$&wenige $|+\rangle$&$<1$       &$<0$       &$von\;-\infty$\\
     $(steigend)$ &viele  $|-\rangle$&$(fallend)$&$(fallend)$&$(steigend)$\\
     \hline     
                  &                  &           &           &\\
     $E_{max}    $&keine  $|+\rangle$&$0$       &$-\infty$  &$0_-$      \\
                  &alle   $|-\rangle$&          &           &            \\ 
     \hline
         
\end{tabular}
\caption{"Ubersicht "uber Zustand und Temperatur eines Systems von $N$ 
unabh"angigen Spins $\frac{1}{2}$ bei verschiedenen Energien}
\label{uebersicht}
\end{center}
\end{table}
Tabelle \ref{uebersicht} beschreibt den Verlauf der Kurve aus Abb. \ref{t}. 

Negative Temperaturen treten in diesem System auf, wenn das Verh"altnis 
von $n_+/n_- < 1$ ist. Die Besetzungen sind in diesem Falle vertauscht. 
Bei $T=0_+$ sind alle Spins im $|+\rangle$ Zustand, w"ahrend bei $T=0_-$
alle Spins sich im $|-\rangle$ Zustand befinden. Obwohl die Differenz zwischen
diesen beiden Temperaturen
klein ist, repr"asentieren sie jedoch v"ollig verschiedene physikalische
Zust"ande. F"ur $T=\infty$ und $T=-\infty$ ist dagegen der Unterschied der
physikalischen Zust"ande gering, obwohl die Temperaturen auf der Skala weit
auseinanderliegen. 
 
\begin{figure} 
\centerline{\epsfig{file=figures/1_durch_t.eps,height=3in}}
\caption{1/T als Funktion von E in einem System von $N$ unabh"angigen Spins 
$\frac{1}{2}$
\label{1t}}
\end{figure}

Tr"agt man dagegen $1/T^*$ "uber $E$ auf, so entspricht diese Kurve viel mehr
der physikalischen Situation. Gro"se physikalische Unterschiede 
liegen auf der Abzisse weit entfernt, w"ahrend geringe Unterschiede nahe
beieinander liegen. Die Kurve ist zudem bei $E=0$ stetig.

\section{Negative Temperaturen}

Wenn die Entropie eine fallende Funktion der Energie ist, treten negative
Temperaturen auf. Diese Situation ist nicht auf das Beispiel des idealen
Paramagneten beschr"ankt, sondern gilt f"ur alle Systeme mit
{\bf endlicher Zahl von Einzelzust"anden}. 
F"ur alle diese Systeme gilt: $E_{min}\leq E \leq E_{max}$. Zur Vereinfachung
kann man nun annnehmen, das der jeweils h"ochste und niedrigste Energiezustand
nur einmal auftreten kann; daraus folgt, da"s $S=0$ sowohl f"ur $E=E_{min}$ als
auch f"ur $E=E_{max}$ f"ur den mikroskopischen Zustand des Systems gilt.

Zwischen $E_{min}$ und $E_{max}$ ist S eine steigende Funktion mit E,
dann eine fallende Funktion,
woraus folgt, das f"ur diesen Bereich $T<0$ ist. 
Damit die Temperatur aber wirklich negativ werden kann, mu"s jedes Element
des betrachteten Systems "uber eine endliche Anzahl von Zust"anden verf"ugen.
Deshalb darf keines der Teilchen "uber kinetische Energie verf"ugen, da dort
das Energiespektrum nicht nach oben beschr"ankt ist.
Selbst Elektronen oder Kerne besitzen jedoch einen Spin, so da"s es scheint,
da"s kein reelles System diese Bedingung erf"ullen kann.
Wir werden sp"ater sehen, da"s bei gewissen Spinsystemen, die "uber eine
gen"ugend lange Relaxationszeit verf"ugen, es jedoch m"oglich ist,
die Freiheitsgerade durch eine negative Temperatur zu beschreiben.

\subsection{Negative Temperaturen sind ``hei"ser'' als positive}

F"ur den Begriff ``hei"ser'' und ``k"alter'' gibt es alternative m"ogliche
Definitionen, welche f"ur positive Temperaturen in Einklang miteinander
stehen, f"ur negative jedoch nicht zutreffen.
So k"onnte man sagen, da"s derjenige K"orper der hei"sere ist, dessen 
Temperatur T den gr"o"seren algebraischen Wert hat. In diesem Falle w"aren
alle positiven T hei"ser als negative T.
Wenn ein K"orper mit positivem T und negativem T miteinander in thermischen
Kontakt stehen, w"urde jedoch die W"arme des K"orpers mit negativem T
zu dem mit positven T flie"sen.
Die Definiton, die am besten mit der normalen Bedeutung "ubereinstimmt,
ist dann auch die die, da"s der hei"sere von zwei K"orpern derjenige ist,
dessen W"arme abgegeben wird, wenn zwei K"orper in thermischen Kontakt 
kommen, w"ahrend der k"altere derjenige ist, der W"arme aufnimmt.

Mit dieser Definition ist jede negative Temperatur hei"ser als jede positive,
w"ahrend f"ur zwei Temperaturen mit dem gleichen Vorzeichen der gr"o"sere
algebraische Wert der hei"sere ist.
Die Skala l"auft dann wie folgt von kalten zu warmen Temperaturen
$0_+ ,\; T>0 ,\; +\infty ,\; -\infty , \; T<0 , \; 0_-$.
So liegt dann z.B. $T=500$ K zwischen $+200$ K und $-200$ K.

Man k"onnte versucht sein, diese sehr verwunderliche Anordnung der 
Temperaturskala als Argument gegen die G"ultigkeit von negativen Temperaturen 
anzuf"uhren. Diese K"unstlichkeit dieser Ordnung ist aber eher ein zuf"alliges
Ergebnis der willk"urlichen historischen Definition der Temperatur in
der Thermodynamik.

H"atte man zur Charakterisierung von ``k"alter''-``w"armer'' die
sich aus der statisischen Mechanik anbietende Gr"o"se
$\frac{1}{T} = \frac{\partial S}{\partial E}$ gew"ahlt, so w"are die
Funktion stetig. Noch besser w"are sogar $-\frac{1}{T}$, weil dann der Wert
umso gr"o"ser ist, je hei"ser der K"orper ist.


\section{Spintemperatur}
Das Existieren von negativen Temperaturen ist nicht auf 
Spinsysteme beschr"ankt (zumindest theoretisch).
Wichtig ist lediglich die Forderung nach der Endlichkeit der
Energiezust"ande und damit nach einer endlichen Energie. Aus der Definition 
der Temperatur "uber die Innere Energie
und der Entropie ist leicht zu sehen, das f"ur jedes System, welches eine 
fallende Entropiefunktion "uber die Innere Energie aufweist, 
die entsprechenden Temperaturen negativ sind.\\
Die Spinsysteme, insbesondere die mit dem Spin $\frac{1}{2}$, sind jedoch
wegen ihrer endlichen Energieniveaus f"ur die Betrachtung negativen 
Temperaturen pr"adestiniert. Gasmolek"ule z.B. haben aufgrund ihrer kinetischen 
Energie ein kontinuierliches Energiespektrum, ohne obere Grenze. Diese Systeme 
sind f"ur die Erzeugung von negativer Temperatur ungeeignet, da man unendlich 
viel Energie ben"otigen w"urde.
Aus der Gleichung
\begin{equation}
\label{spintemp}
\frac{1}{T}= \frac{k}{2\mu B}\; ln \frac{n_{+}}{n_{-}}
\end{equation}
sieht man leicht, da"s f"ur Systeme mit endlicher Energie eine Temperatur 
"uber  die Besetzungzahlen der einzelnen Zust"ande definierbar ist.
Experimentell bedeutet die Forderung der endlichen Energie ein Fehlen der 
kinetischen Energie. 
Das System mu"s also vollkommen station"ar sein, um die obige Bedingung 
erf"ullen zu k"onnen. \\
Wie schon oben beschrieben, besteht unser System aus den zwei entkoppelten 
Systemen des Spins und des Gitters. Die einzigen Zust"ande des Spinsystems
sind $|+\rangle$ und $|-\rangle$ und damit sind die zugeh"origen Energien ebenfalls
endlich. Will man jetzt die Temperatur des Spinsystems bestimmen, so mu"s
man dieses von dem Gittersystem als vollkommen entkoppelt betrachten. 
Experimentell ist dies bei geeigneten Stoffen genau dann m"oglich, wenn die
Relaxationszeit des Spinsystems klein gegen"uber der Spin-Gitter-Relaxationszeit 
ist. F"ur Experimente dieser Art eignet sich Lithiumfluorid besonders, da 
hier Spin-Gitterrelaxationszeiten $\tau$ bis zu 5 min betragen k"onnen
und damit gro"s gegen"uber den Spinrelaxationszeiten $\tau_{S}$ mit weniger
als $10^{-5}$ s sind. Innerhalb des Zeitraumes $\tau$ findet zwischen dem 
Spinsystem und dem Gittersystem praktisch kein Energieaustausch statt. 
Die Systeme sind innerhalb des Zeitraumes $\tau$ adiabatisch isoliert.\\
Innerhalb des Zeitraumes $\tau$ kann sich das Spinsystem also mehrmals in
einen makroskopischen Gleichgewichtszustand einpegeln, so da"s die Temperatur
des Spinsystems unterschiedlich von der Gittertemperatur sein kann. Der
makroskopische Gleichgewichtszustand ist aber zudem notwendig, um "uberhaupt 
von einer Temperatur des Systems reden zu k"onnen. Die Temperatur des 
Gittersystem ist hierbei immer positiv, w"ahrend die Spintemperatur positiv 
oder negativ sein kann.


\subsection{"Anderung der Spintemperatur}

Aus der Definition der Spintemperatur in der Gleichung \ref{spintemp}
lassen sich zwei M"oglichkeiten entnehmen die Spintemperatur zu beeinflussen:
\begin{itemize}
\item Ver"anderung der Verh"altnisses $\frac{n_{+}}{n_{-}}$ bei konstantem 
Magnetfeld
\item Ver"anderung des Magnetfeldes bei konstantem Verh"altnis der 
Besetzungszahlen
\end{itemize}


Die Ver"anderung der Besetzungszahlen l"a"st sich durch das Einstrahlen 
einer elektromagnetischen Sinuswelle mit der Frequenz
$\sumofstates{0}{}=\frac{1}{\bar h} \mu B$ erreichen. Dies ist jedoch nur dann
m"oglich, wenn vor dem Einstrahlen schon $n_{+}>n_{-}$ galt. Es wird in 
diesem Fall mehr absorbiert als emittiert und somit sinkt das 
Besetzungsverh"altnis $\frac{n_{+}}{n_{-}}$. F"ur $n_{+}=n_{-}$ sind 
Absorption und Emission gleich gro"s und damit bleibt auch das 
Besetzungverh"altnis konstant. Das hat zur Folge, da"s durch Einstrahlung
einer elektromagnetischen Welle zwar eine sehr hohe Spintemperatur 
erreichbar ist, der "Ubergang zur negativen Energie jedoch nicht gelingt. 
Andererseits ist es sehr einfach das "au"sere Magnetfeld zu modifizieren,
solange es nicht das Magnetfeld der Erde ist. Eine Variation des Feldes 
in einem Zeitraum der klein gegen die Spin-Gitter-Relaxationszeit ist,
l"a"st die Spinzust"ande unver"andert. Aus der Gleichung \ref{spintemp}
sieht man, da"s sich  die Temperatur des Spinsystems proportional zu
dem angelegtem Magnetfeld "andert. Verkleinert man die St"arke des 
Magnetfeldes, so sinkt auch die Spintemperatur. Dieses Ph"anomen ist 
unter {\em Abk"uhlung durch adiabatische Entmagnetisierung} bekannt. 

Ist das Magnetfeld in dem der Kristall liegt start genug, so richten sich 
die atomaren Magneten parallel zum Magnetfeld aus. Wird das Magnetfeld dann
pl"otzlich abgestellt, so bewirkt die Zitterbewegung der Gitteratome, da"s 
die Ausrichtung der Magnete wieder zuf"allig wird. Diese Ausrichtung 
ben"otigt jedoch keinerlei Energie, so da"s die Temperatur des Kristalls 
unver"andert bleibt. Um den Kristall abzuk"uhlen wird das Magnetfeld
deshalb nicht abgeschaltet, sondern nur soweit geschw"acht, da"s die 
Ausrichtung der Magneten gegen das Magnetfeld m"oglich wird. Diese
Ausrichtung geschieht gegen das Magnetfeld, an dem so Arbeit geleistet
wird. Die dazu ben"otigte Energie wird der thermischen Energie des 
ristalls entzogen, so da"s die Kristalltemperatur sinkt.\\
   
Aber auch dieser Vorgang erlaubt keine negativen Temperaturen des Systems. 
Zur Erzeugung von negativen Temperaturen geht man daher wie folgt vor:
\begin{itemize}
\item Anfangszustand mit einem relativ starken Magnetfeld um 6000 Gau"s
\item adiabatische Entmagnetisierung auf 100 Gau"s
\item Umkehrung der Feldrichtung innerhalb einer sehr kleinen Zeitspanne 
$t < 10^{-5}$ s
\item adiabatische Magnetisierung auf die urspr"unglichen 6000 Gau"s
\end{itemize}
Die eigentliche Erzeugung der negativen Temperaturen geschieht bei der
Umkehrung der Richtung des Magnetfeldes innerhalb eines kurzen Zeitraumes. 
Diese Umkehrung geschieht so schnell, da"s die einzelnen Spins keine Zeit 
haben ihren Zustand zu "andern. Effektiv werden durch das Wechseln der 
Magnetfeldrichtung jedoch die Zust"ande $|+\rangle$ und $|-\rangle$ vertauscht. 
Wenn das Verh"altnis der Besetzungszahlen 
$\frac{n_{+}}{n_{-}}$ vorher gr"o"ser 1 war, d.h. das System eine positive 
Spintemperatur aufwies, so ist das Verh"altnis nach dem Wechsel reziprok und 
damit kleiner 1. Die Temperatur des Spinsystems ist somit per Definition 
negativ. Ihre Lebensdauer entspricht der Spin-Gitter-Relaxationszeit $\tau$.
Innerhalb dieser Zeit f"uhrt das Spinsystems dem Gittersystem W"arme zu. 
Die negativen Temperaturen sind also immer w"armer als die positiven 
Spintemperaturen, da das Systeme mit negativer Temperatur st"andig Energie 
an das Gittersystem abf"uhrt.
Die Erhaltung von negativen Temperaturen gelang zuerst Purcell und Pound, 
die diese "uber mehrere Minuten aufrecht erhalten konnten.
Der experimentelle Beweis f"ur die Existenz von negativen Spintemperaturen 
gelang, indem man die Besetzungzahlen der Spinzust"ande mittels 
{\em magnetischer Kernresonanz } bestimmte. \\
Nimmt man eine Probe eines Kristalls, wie z.B. Lithiumfluorid, und legt diese 
in eine Spule, die ein magnetisches Feld erzeugt. Dieses Feld oszilliert mit 
der Frequenz $\sumofstates{0}{}$. Die Energie des Feldes wird f"ur 
$\sumofstates{0}{}=\frac{1}{\hbar}\mu B$ und $\sumofstates{0}{}=\frac{1}{\hbar}\mu B$ 
Absorption, bzw Emission induzieren. Bei der Emission klappt ein Gitteratom 
aus dem $|-\rangle$-Zustand in den $|+\rangle$-Zustand um, und emittiert dabei ein Photon 
der entsprechenden Energie. Bei der Absorption wird die Energie $\mu B$ 
aufgenommen, die wiederum der Spule entzogen wird. Durch die Energieverlust 
an der Spule l"a"st sich so auf die Besetzungzahlen der einzelnen Zust"ande 
r"uckschlie"sen.
Ist "uber die Messung erst bewiesen, da"s Spin und Gitter verschiedene 
Temperaturen annehmen k"onnen, so folgt automatisch die Existenz der negativen 
Spintemperaturen.
\end{module}

\section{Literatur}
1. R.V. Pound, Phys. Rev. {\bf 81}, 156 (1951) \\
2. E.M. Purcell und R.V. Pound, Phys. Rev. {\bf 81}, 278 (1951) \\ 
3. E.M. Purcell und R.V. Pound, Phys. {\bf Rev. 81}, 279 (1951) \\ 
4. Norman F. Ramsey, Phys. Rev. {\bf 103}, (1956) \\
5. A. Abragam, Phys. Rev. {\bf 109}, 1441 (1958) \\
6. F. Reif: Statistische Physik und Theorie der W"arme, de Gruyter 1987 \\
7. Diu, Guthmann: Grundlagen der Statistischen Physik, de Gruyter 1994 \\

\end{document}
  


