\def\llquote#1{\ensuremath{\langle\kern-.25em\langle{#1}\rangle\kern-.25em\rangle}}
\def\PHYSmodule#1{\module{PHYS}{#1}{Foundations of Physics}} 
\def\mobjabbr{OMOBJ |m:math |legacy}
\def\module#1#2#3{#1}
\def\indextoo#1{#1}
\def\snippet#1{\tt{#1}}
\def\Kelvin{K}
\def\Celsius{{}^{\circ}C}

\section{Extending OMDoc for Physics}\label{HandleOnKnowledge}

In this section we will extend the {\omdoc} format\footnote{Due to space restrictions we
  cannot introduce the format here; we refer the reader to~\cite{Kohlhase:omdoc1.2} for
  the language definition and examples.} by an infrastructure for (physical) systems,
observables and experiments.
\begin{figure}
\begin{center}\scriptsize
\begin{tabular}{|>{\tt}l|>{\tt}l|>{\tt}l|>{\tt}p{5.5cm}|}\hline
{\rm Element}& \multicolumn{2}{l|}{Attributes} & Content  \\\hline
             & {\rm Req.}  & {\rm Optional}     &           \\\hline\hline
 observable  & name & algebra, xref & metadata?, opdef, refinement, type?\\\hline
 refinement  &      & xml:id, xref & metadata?, CMP*, FMP*\\\hline 
 opdef       &      & xml:id, xerf & metadata?, CMP*, FMP*\\\hline\hline
 system      &     & xml:id, xref & metadata?, realization?, observable*, 
                                               preparation?, state?\\\hline
 realization &     & xml:id & metadata?, CMP*, FMP*\\\hline 
 preparation &     & xml:id & metadata?, CMP*, FMP*\\\hline 
 state       & of  & xml:id, xref & metadata?, value*\\\hline
 value       & for & xml:id, xref & \llquote{mobj}\\\hline\hline
 experiment  &                 & xml:id, xref   & metadata?, CMP*, FMP*, measurement*\\\hline
 measurement &           & xml:id, xref   & metadata?, state, state \\\hline
 evidence    & for, type & xml:id, & metadata? CMP*, FMP*,interpretation \\\hline
 interpretation &        & xml:id  & metadata?, CMP*, FMP*\\\hline 
 \multicolumn{4}{|p{11cm}|}{where {\element{metadata}}, {\element{CMP}}, {\element{FMP} and
     {\element{type}} are {\omdoc} elements described in~\cite{Kohlhase:omdoc1.2} and where \llquote{mobj} is {\tt{(\mobjabbr)}}}}\\\hline
\end{tabular}\vspace*{-.5cm}
\end{center}\normalsize
\caption{The Structure of {\physml} Elements}\label{fig:overview}\vspace*{-.5cm}
\end{figure}

With the existing representational infrastructure in {\omdoc} we can already represent
structured collections of interrelated concepts and statements about them via {\omdoc}
{\element{theory}}\footnote{The nomenclature in mathematics, which gave rise to the
  element names in {\omdoc} and the naming conventions in physics clash here. In physics a
  set of assumptions about the physical world are called a ``model'' until they are
  generally accepted, only then are they called a ``theory'' (e.g.  the Nuclear
  shell-model; however: quantum theory, general relativity theory).  } contexts. One of
the central concepts in physics, the theory of measurable quantities can be set up in this
way using {\omdoc} {\element{symbol}}s.

We start with a simple example, the dimensions of the SI units.

\begin{lstlisting}[label=lst:concepts,mathescape,
  caption={Introducing Basic Concepts in a {\omdoc} Theory},
  index={theory,symbol}]
<theory xml:id="dimensions">
  <symbol name="mass"/><symbol name="length"/><symbol name="time"/>
  <symbol name="charge"/><symbol name="temperature"/>
  <symbol name="volume"/>
  <definition for="volume" type="simple">$\om{length^3}$</definition>
</theory>
\end{lstlisting}
We can introduce derived dimensions like the dimension for volume as defined
concepts. Note that all of the {\element{symbol}} declarations make the concepts available
for the use in {\openmath}-encoded formulae via {\element{OMS}} elements and for the
markup of technical terms via the {\omdoc} {\element{term}} element. Both identify a
concept by its {\attribute{name}{OMS}} and home theory (called a {\emph{content
    dictionary}}; hence the attribute {\attribute{cd}{OMS}}). Here as in the following, we
use mathematical notation in boxes to abbreviate the {\openmath} objects in the listings
to save space.

We will use these dimensions as a type system for quantities, and introduce the units as
constructors for the dimensions (note that we introduce the symbols with a
{\element{type}}\footnote{In the example, we have not executed this, but it is possible
  to extend the type system to model ranges of numerical values in quantities in this type
  system: Instead of simply specifying that the unit $\Kelvin$ is of type
  {\tt{$\backslash$temperature}} we give $\Kelvin$ the complex type $\langle
  temperature,{\mathbb{R}}^*\rangle$ and adjust the dimension-types of the arithmetic
  operators, so that they check for range admissibility. This puts a considerably higher
  load on the type checking algorithm, but gives more control and quality assurance. As
  {\omdoc} encoding tolerates multiple type systems, we need not even choose one, but
  can accumulate the knowledge in the representations and use the one appropriate to the
  task at hand.}).
\begin{lstlisting}[label=lst:units,mathescape,caption={A Theory of SI Units},
  index={theory,symbol}]
<theory xml:id="units">
  <symbol name="gram"><type system="dimensions">$\om{mass}$</type></symbol>      
  <symbol name="Kelvin"><type system="dimensions">$\om{temperature}$</type></symbol>
  <presentation for="Kelvin">$\Kelvin$</presentation>
  <symbol name="Celsius"><type system="dimensions">$\om{temperature}$</type></symbol>
  <presentation for="Celsius">$\Celsius$</presentation>
  <definition for="Celsius" type="implicit">$\om{\forall x\not=0. \Kelvin=(x-273.15)\Celsius}$</definition>
</theory>
\end{lstlisting}
As usual, we can define the intended notation of a concept via {\element{presentation}}
elements (see section~\ref{sec:physml-stex}) and we can introduce derived units via
definitions. With this machinery, we can also state natural laws:
\begin{lstlisting}[label=lst:nat_law,mathescape,
  caption={A Natural Law Expressed as an {\omdoc} {\element{axiom}}.},
  index={axiom}]
<axiom xml:id="force_mass_acceleration" type="natural_law">
  <CMP>Force is mass times acceleration.</CMP>
  <FMP>$\om{{\bf{F}}={\bf{m}}\cdot {\bf{a}}}$</FMP>
</axiom>
\end{lstlisting}
Note that in {\omdoc} terminology we are dealing with an axiom, i.e. with an assertion
that cannot be mathematically proven\footnote{There may be physical evidence that supports
  it though.} but has to be assumed about the world.  In physics a relation between
observables has to be supported by sets of experiments, with no counter-evidence within the
range of the variables of the involved observables.

\subsection{Observables}\label{subsec:observables}

Above we have determined the notion of an {\emph{observable}} as a primary object of
physics. As any observable --- e.g. the temperature, or velocity --- of a given physical
system can be used in formulae describing the system, we need to extend the {\omdoc}
format by a new statement-level language element that is definition-like. The
{\eldef{observable}} element introduced by the {\physml} module in {\omdoc} (see
Figure~\ref{fig:overview} for an overview) has three relevant children\footnote{Here and
  in the following, we will not explicitly describe the {\element{metadata}} element,
  which is used in {\omdoc} to accommodate bibliographic and administrative metadata,
  specifying titles, digital rights, licensing, authorship, timestamping, etc. or the
  {\attributeshort[ns-attr=xml]{id}} attribute which is used for identification. Details
  can be found in the {\omdoc} specification~\cite{Kohlhase:omdoc1.2}.} {\element{opdef}},
{\element{refinement}}, and {\element{type}}, to model the properties of observables we
have identified in Section~\ref{sec:physml}. The {\eldef{opdef}} and {\eldef{refinement}}
elements contain {\emph{mathematical vernacular}}, i.e. structured text interspersed with
mathematical formulae. Mathematical vernacular is represented in {\omdoc} by a
multilingual group of {\element{CMP}} (commented mathematical property) elements with
mathematical text, and (possibly) a multi-system group of {\element{FMP}} elements with
formalizations of the properties expressed in the {\element{CMP}}s. The {\element{opdef}}
element is used for describing the {\emph{operational definition}} of the observable,
i.e. the defining process of measurement, whereas the {\element{refinement}} element is
used to specify the rule of iterative refinement that takes the measurement process to
its (idealized) limit.

The {\emph{dimension}} of the observable is specified as a {\element{type}} element. Here
we can directly use the type system for dimensions we have introduced in the last section.
In our example in Listing~\ref{lst:observable} this is just the temperature.

The {\element{observable}} element carries a {\attribute{name}{observable}} attribute,
which is used by {\omdoc} to introduce a symbol that can be referenced by an
{\element{OMS}} element just like the {\element{symbol}} element. Furthermore, it carries
an optional {\attribute{algebra}{observable}} attribute that contains a pointer to an
{\omdoc} representation to the mathematical object introduced by this observable.

All of these elements also carry an optional {\attributeshort{xref}} attribute that allows
to refer to an already existing representation of the same element via an URI reference;
the effect is that the referred object is virtually copied in to the place of the
referring one.

\begin{lstlisting}[label=lst:observable,mathescape,
  caption={An Observable for the Temperature},
  index={observable,opdef,refinement,type}]
<observable name="temperature">
  <opdef><CMP>Measure with a thermometer.</CMP></opdef>
  <refinement><CMP>Make the thermometer stepwise smaller.</CMP></refinement>
  <type system="dimensions">$\om{temperature}$</type>
</observable>
<presentation for="temperature"><use format="default">$\bf T$</use></presentation>
\end{lstlisting}


\subsection{Physical Systems and their States}\label{subsec:systems}

One of the basic building block of {\physml} is the {\eldef{system}} element that is used
to represent a physical system. The system is described via the mathematical vernacular in
a {\eldef{realization}} element which is the first relevant child. As we have seen above,
a physical system can be characterized by a (in practice very finite) set of
{\indextoo{observable}s}, i.e. physical variables that can be measured independently.
These are represented by a non-empty set\footnote{Enjoy the special cases: By use of an
  apparatus, which cannot measure anything (that is: has no observable) one cannot learn
  anything. The respective mathematical operator would be the identity.  Less trivial is
  the case, where we prepare a system in state $|a\rangle$, then try a measurement `is the
  system in state $|a\rangle$'? If it is already in that state, one does not learn
  anything new, and that is: no-one can decide whether the experiment took place or not.
  Example: heat a system and a thermal measuring device to 40 deg.  Then measure the
  temperature of the system by the device: Your result 40 deg can by no means be
  distinguished from the suspicion you did not do the experiment.}  of
{\element{observable}} children. Listing~\ref{lst:simple-system} shows a very simple
system, which we will use as a concrete measuring apparatus later.

\begin{lstlisting}[label=lst:simple-system,mathescape,
  caption={A Simple Physical System},
  index={theory,symbol}]
<system xml:id="thermometer">
  <realization><CMP>A thin glass tube with mercury in it.</CMP></realization>
  <observable xref="#temperature"/>
</system>
\end{lstlisting}

In this setup, we have represented only the observable we are interested in: all other
physical traits of the apparatus are irrelevant for our current purposes. If other
physical properties also matter, then we can add other observables. However, we have to
make sure that we fix the states of all of the observables that we do not want to
measure. This can be done informally in mathematical vernacular in the optional
{\eldef{preparation}} element, which may follow the {\element{observable}} elements, and
more formally in a {\element{state}} element. A {\eldef{state}} element specifies a set of
values for observables in the system it refers to (either its parent {\element{system}} or
the system specified to in the optional {\attribute{of}{state}} attribute) via a set of
{\element{value}} children. A {\eldef{value}} element specifies the observable it refers
to by referring to it's name in the required {\attribute{for}{value}} attribute. Its
content is a representation of a physical quantity as an {\openmath}, content {\mathml},
or {\omdoc} {\element{legacy}} element. In the example below, we have (somewhat
arbitrarily) prepared a gas cylinder for an experiment by making it red.

\begin{lstlisting}[label=lst:system,mathescape,
  caption={A Physical System Prepared for an Experiment},
  index={theory,symbol}]
<system xml:id="gas_cylinder">
  <realization><CMP>A gas-tight wooden cylinder</CMP></realization>
  <observable xref="#pressure.obs"/>
  <observable xref="#density.obs"/>
  <observable xref="#color.obs"/>
  <preparation><CMP>We make the cylinder red!</CMP></preparation>
  <state><value for="color">$\om{red}$</value></value>
  </state>
</system>
\end{lstlisting}

\subsection{Experiments}\label{subsec:experiments}

Physical experiments are represented by the {\element{experiment}} element in
{\physml}. The body of this element consists of  two {\element{system}} elements followed by a
set\footnote{We explicitly allow an empty set of measurements here in order to describe
  future, planned or failed experiments that have not yielded measurements (yet).} of
{\element{measurement}} elements. The first child represents the system which is measured,
the second the measuring device. The {\eldef{measurement}} elements contain two
{\element{state}} elements as described above which correlate the state of the system on
which the measurement is performed with the state of the system of the measuring
device. In the following example, we represent the result of measuring the temperature of
a gas cylinder with varying density and pressure.  
\begin{lstlisting}[label=lst:experiment,caption={Experiment: measuring the temp.
    of a gas cylinder},
  index={state,value},mathescape]
<experiment xml:id="ex_pressure_vs_temp">
  <CMP>Measuring the pressure vs. temperature of a compressed gas cylinder</CMP>
  <system xref="#gas_cylinder"/>
  <system xref="#thermometer"/>
  <measurement xml:id="m_213">
    <state of="#gas_cylinder">
      <value for="pressure">$\om{332.49586 psi}$</value>
      <value for="density">$\om{19 g/l}$</value>
    </state>
    <state of="#thermometer"><value for="temperature">$\om{17.52\Kelvin}$</value></state>
  </measurement>
</experiment>
\end{lstlisting}
Note that this only represents the raw data from an experiment. We can link experiments
and natural laws, such as the one stated in {\mylstref{nat_law}} via the
{\element{evidence}} element. The main insight here is that as we cannot ``prove'' natural
laws, but only observe them. We can only keep on experimenting in physics and collect
evidence or counter-evidence for any relations between observables. The {\eldef{evidence}}
element contains a non-empty set of {\element{experiment}}s followed by an
{\element{interpretation}} element that allows to detail any interpretative steps, e.g. an
account how the data was fitted to a curve, etc. Its {\attribute{for}{evidence}} attribute
specifies the relation it concerns, and the {\attribute{type}{evidence}} attribute
specifies whether the evidence supports it (value {\tt{for}}) or falsifies it (value
{\tt{against}}).

In reality one is left with a residual ambiguity because physical experiments are
conducted with real apparata, while the physics law gives a mathematical relation between
the idealized quantities of the physical observables and apparata obtained as the
(virtual) limit of the stepwise refinement iteration rule.

%%% Local Variables: 
%%% mode: LaTeX
%%% TeX-master: "mkm06"
%%% End: 

% LocalWords:  em PHYS mobj num Ebs stimmt das XXX YYY bitte korrigieren lst cd
% LocalWords:  mathescape OMF arith dec nat MiKo ns attr Heinrich hier vern bzw
% LocalWords:  unftige Werte Observablen einsetzen dc kmp xref mkm OMOBJ truecm
% LocalWords:  xml metadata CMP FMP Ich glaube nein aber nachdenken reinkoennte
% LocalWords:  IEEE OMA multi Req opdef xerf stex deg obs
