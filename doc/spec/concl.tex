\section{Conclusion and Further Work}\label{sec:conclusion}

We have demonstrated that a Markup Language for the {\emph{content}} of physics can be
designed by extending the content and context markup format {\omdoc} with a
representational infrastructure for the principal objects of physics: observables,
systems, and experiments. The resulting language {\physml} is able to catch the logical
and operational structure specific to physics, differentiating this field from others. The
extension presented in this paper is part of the ongoing enterprise to extend the {\omdoc}
format to the {\bf{STEM}} fields ({\underline{S}}ciences, {\underline{T}}echnology,
{\underline{E}}ngineering and {\underline{M}}athematics).


The next step is now to evaluate the language by marking up a larger body of knowledge in
physics in {\physml}.  We have started work on the technically ubiquitous and basic field
of thermostatics. This should give us a clear indication whether {\physml} is adequate for
all of physics, or pinpoint the necessary changes to the language design.  An
international collaboration on the further development of {\physml} is looked for,
including experts from theoretical and applied physics and related fields, in particular
mathematics and chemistry.

New and powerful services can be implemented once the scientific content can be
semantically encoded, retrieved, and reused digitally.  In physics, these include the
search for other experiments on the same observables, dimension and algebraic checking of
mathematical equations, mapping to other mathematical representations of the same
theoretical physical expression, etc.

Using the approach of analyzing the operational and logical practices of a scientific
discipline field, and map this to field-specific modules extending the semantic markup
language {\omdoc} will allow to spread semantic content markup to other scientific fields.

With authors to increasingly make use of markup languages, and retrieval engines
following suit to offer intelligent search algorithms making use of the known markup
languages, users will gain effective tools to increase the reachout of their scientific work,
having the {\emph{content}}, not just the text, of the work of others at their fingertips.

%%% Local Variables: 
%%% mode: LaTeX
%%% TeX-master: "mkm06"
%%% End: 

% LocalWords:  mkm
