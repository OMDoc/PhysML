\documentclass{mikoslides}
\usepackage{times,alltt}%,url,amssymb}
\usepackage{acronyms,logics2e,varia}% from ~/tex/lib/mymacros
\usepackage{paths,stex,mikoaffiliation}% from sTeX
\usepackage{tikz}
\def\latexml{\scsys{LaTeXML}}

\begin{document}\sf
\begin{slide}\strut
  \begin{ttitle}
   \textcolor{red}{Capturing the Content of Physics}\\
  \end{ttitle}\vspace{1cm}
  \begin{center}
  {\color{blue}\miko}
  \end{center}
\end{slide}

\input{\omdocsnippath{scientific-method}}
 
\begin{separatorslide}{}
  \begin{ttitle}
   \color{red} Content Markup Formalisms\\x1
   \begin{minipage}{8.5cm}\normalsize
   \begin{itemize}
   \item Mathematical Formulae ({\mathml}, {\openmath})
   \item Program Fragments ({\codeml})
   \item Molecules in Chemistry ({\chemml})
   \item {\textcolor{red}{Observables, Systems, Experiments ({\scshape{PhysML}})}}
   \end{itemize}
   \end{minipage}
  \end{ttitle}
\end{separatorslide}

\input{\mathmlsnippath{overview}}
\input{\mathmlsnippath{layout-schemata}}
\input{\omsnippath{nutshell}}
\input{\mathmlsnippath{parallel}}

 \begin{separatorslide}{Markup for Structured Scientific Documents}
   {\omdoc}: Open Mathematical Documents\\
   Extend this to cover (Computer) Science materials
 \end{separatorslide}

\input{\omdocsnippath{situating}}
\input{\omdocsnippath{nutshell}}
\input{\omdocsnippath{statements}}
\input{\omdocsnippath{definition}}

\input{\omdocsnippath{presentation}}
\input{\omdocsnippath{physml-units}}
\input{\omdocsnippath{physml-observables}}
\input{\omdocsnippath{physml-systems}}
\input{\omdocsnippath{physml-experiment}}
\input{\omdocsnippath{activedocs}}
\input{\omdocsnippath{physml-stex}}

\begin{frame}
  \frametitle{Conclusions}
  \begin{itemize}
  \item Towards an E-Science {\textcolor{green}{infrastructure}} based on
    {\textcolor{green}{communication}} of {\textcolor{green}{knowledge}}
  \item {\textcolor{blue}{Approach}:} Start with {\textcolor{red}{content/context markup}}
    for {\textcolor{green}{math}} and {\textcolor{red}{physics}}.
  \item {\color{red}\omdoc} standardized communication language\lec{based on {\xml}}
  \item gives us the beginning of a {\color{green} E-Science assistant}
    \begin{itemize}
    \item integration of external software systems\lec{make use of the Internet}
    \item a common ontology for integration\lec{is zero a natural number?}
    \item a universal math/-Science repository\lec{data mining, semantic search}
    \item bookkeeping in experiments and theories\lec{management of change}
    \end{itemize}
  \item {\textcolor{blue}{Slogan}:} Light-weight formal methods for added-value applications
  \end{itemize}
\end{frame}

\input{\omdocsnippath{theory-nutshell}}
\input{\omdocsnippath{theory-natlist}}
\input{\omdocsnippath{theory-inclusion}}


\end{document}
%%% Local Variables: 
%%% mode: latex
%%% TeX-master: t
%%% End: 

% LocalWords:  ISN Hilf def Baumann Unitan AutoMath qed NT physml activedocs
% LocalWords:  natlist
