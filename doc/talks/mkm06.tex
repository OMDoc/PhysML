\documentclass{mikoslides}
\usepackage{times,amssymb,alltt}
\usepackage{calbf}% from ~/tex/lib/mymacros
\usepackage{paths,stex,mikoaffiliation}% from sTeX
\usepackage{tikz}
\def\latexml{\scsys{LaTeXML}}

\begin{document}\sf
\begin{slide}\strut
  \begin{ttitle}
   \red{Capturing the Content of Physics}\\
   \red{Systems, Observables, and Experiments}
  \end{ttitle}\vspace{1cm}
  \begin{center}
  {\color{blue}\mikoand{Eberhard Hilf (Oldenburg) \& Heinrich Stamerjohanns (IUB)}}
  \end{center}
\end{slide}

\begin{frame}
  \frametitle{Goals and Results of this work}
  \begin{itemize}
  \item {\red{From a MKM prespective}}\lec{my angle}
    \begin{itemize}
    \item {\blue{Goal}:} Validate MKM technology by extending its scope
    \item {\blue{Goal}:} fulfil the ``Math as a test-tube'' promise
    \item {\blue{Result}:} content/context markup scales for cross-domain
      interoperability!
    \end{itemize}
  \item {\red{From a Physicists perspective}}: Participate in MKM technology
    \begin{itemize}
    \item {\blue{Goal}:} deep structure, searchability, interoperabiliy,
      long-term archiving
    \item {\blue{Goal}:} Active Documents: click-through access to experimental
      data in published documents
    \item {\blue{Result}:} Markup Language for Physics
    \end{itemize}
  \end{itemize}
\end{frame}

\input{\physmlsnippath{desiderata}}
\input{\physmlsnippath{science-of-measurements}}
\input{\physmlsnippath{principal-objects}}

\input{\physmlsnippath{nutshell}}


\input{\physmlsnippath{units}}
\input{\physmlsnippath{observables}}
\input{\physmlsnippath{physml-natural-laws}}
\input{\physmlsnippath{systems}}
\input{\physmlsnippath{physml-experiment}}

\input{\physmlsnippath{stex}}

\input{\omdocsnippath{what-is-escience}}
\input{\omdocsnippath{nutshell}}
\input{\omdocsnippath{omdoc-book}}
\input{\omdocsnippath{scientific-method}}
\input{\omdocsnippath{chemml-nutshell}}
\input{\omdocsnippath{omdoc-raster}}
\input{\omdocsnippath{activedocs}}

\begin{frame}
  \frametitle{Conclusions}
  \begin{itemize}
  \item Towards an E-Science {\green{infrastructure}} based on
    {\green{communication}} of {\green{knowledge}}
  \item {\blue{Approach}:} Start with {\red{content/context markup}}
    for {\green{math}} and {\red{physics}}.
  \item {\color{red}OMDoc} standardized communication language\lec{based on XML}
  \item gives us the beginning of a {\color{green} E-Science assistant}
    \begin{itemize}
    \item integration of external software systems\lec{make use of the Internet}
    \item a common ontology for integration\lec{is zero a natural number?}
    \item a universal math/-Science repository\lec{data mining, semantic search}
    \item bookkeeping in experiments and theories\lec{management of change}
    \end{itemize}
  \item {\blue{Slogan}:} Light-weight formal methods for added-value applications
  \end{itemize}
\end{frame}


\end{document}
%%% Local Variables: 
%%% mode: latex
%%% TeX-master: t
%%% End: 

% LocalWords:  ISN Hilf def Baumann Unitan AutoMath qed NT physml activedocs
% LocalWords:  natlist
