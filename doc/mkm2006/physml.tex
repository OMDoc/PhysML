\section{Desiderata for a Physics Markup Language}\label{sec:physml}

The design of a semantic markup language for a learned field is more sophisticated than it
might seem.  The reason is that, in order to be useful it has to map the way research is
organized. This leads to language designs centered around the {\emph{principal objects}}
of the respective research field. In chemistry, the Chemical Markup Language
{\chemml}~\cite{CML:web} was designed with the {\emph{name of a molecule}} as principal
object, to which properties and its chemical reactions are attached as
properties\footnote{We think that an alternative approach would have had more merits, to
  designate {\emph{chemical reactions}} as the principal objects --- the principal action,
  a chemist does.}.

In mathematics, the earliest discipline to have a dedicated markup language, we have
{\mathml} and {\openmath} as markup formalisms that take mathema\-tical objects as
principal objects. The {\omdoc} format~\cite{Kohlhase:omdoc1.2} extends them with content
markup for {\emph{statements}} (like definitions, axioms, theorems, and proofs) and
{\emph{theories}} (conceived as principal objects on a higher level).

In physics, since the times of Galilei (`\emph{Experimental results are the highest
  authority}'), the young Einstein (`\emph{A theory is to be accepted if it describes and
  predicts all possible experiments,--- independent of the feelings and `intuition' of the
  scientist}'~\cite{Born:einstein}), and most important of P.
Bridgeman~\cite{Hilf:bridgeman}, who elucidated first the logical steps of physics
learning, by analyzing the {\emph{operational}} steps in doing research, it is now
consensus~\cite{Mittelstaedt:buch,Falk:thermodyn,Falk:mechanik,Sakurai} that research is
accepted as physics if it does physical experiments with apparatuses which represent
physical observables.  This simple sounding requirement will be the entry point for us to
design a specific Physics Markup Language, a construction which tries to mirror the way
physicists operationally think.

\subsection{Physics as a Science of Measurements}

We start the same way as a physicist would enter a new field: by {\emph{operationally}}
following the consecutive steps:
\begin{description}
\item[DP1] Decide to work on a specific field, and gather {\emph{'pre-scientific'
      available knowledge}}\footnote{By this we concentrate on fields of interest, where
    we at least {\emph{assume}} that by preparing physical experiments we may gain new
    knowledge.}
\item[DP2] Define an {\emph{observable}}; In physics this has to be done by
  {\emph{constructing a physical device}}\footnote{Historically, in the 1960s, there
    has been a long debate, whether in classical Mechanics, in contrast to all other
    fields, instead of {\emph{building}} a physical device, already the
    {\emph{description}} of how to build that device is
    sufficient~\cite{Mittelstaedt:buch}}, called an apparatus, with which measurements can
  be made giving real valued numbers\footnote{In the modern {\emph{Theory of
        Measurements}} observed numbers are to be mapped to {\emph{Eigenstates}} of a
    {\emph{Hermitian Operator}}, which is the mathematical image of the physical
    apparatus.} depending on the specific experimental setup.\footnote{This rule separates
    physics from other fields, such as mathematics.}
  
\item[DP3] {\emph{Set an iterative operational construction rule}} to refine stepwise the
  design of the apparatus such that (just as in `proof by induction') by applying the rule
  iteratively, it will be accepted that successively more precise apparatuses can be built
  in principle\footnote{This absolutely essential rule assures that we stay with doable
    physics experiments.  The condition that the rule has not to depend on the status of
    actual refinement assures that the limit (see next rule) to a virtual ideal
    mathematical counterpart of the observable will be secure and correct.}
  
\item[DP4] {\emph{Take this construction to its limit}}, and define the virtual outcome of
  such an `ideal device' as the {\emph{physical observable}}, which then can directly be
  related (mapped) to the respective mathematics.\footnote{A Hilbert operator for the
    ideal apparatus, a Hilbert State for its actual physical momentary realization, and
    {\emph{eigenvalues}} for its measurement results. We confess that in practice, most
    scientists use real continuous variables for convenience, say for the position in
    space of a classical mechanical object, but with the strategy given here, we assure
    that we arrive at the correct quantum mechanics first and gain the classical mechanics
    statement by averaging over space from there using the standard Ehrenfest principle.
    The price for the convenience is high: we have to use Banach spaces instead of Hilbert
    space, any proof has to be done by iterating back to Hilbert space, use of
    distribution theory instead of functions, etc.} By this mapping to mathematical
  objects and their algebra, theoretical physics can be done with the aim to reproduce all
  previously conducted experiments and correctly predict any doable experiment in the field within the
  construction-dependent uncertainties of the apparatuses.
\item[DP5] {\emph{Do a set of experiments, map to theory, check with the predictions}} ---
  if all are borne out we have a new natural law (Otherwise the set of assumptions and
  results are called `{\emph{model}}').
\item[DP6] {\emph{Distribute the results in a way so that the experiments and calculations
      can be repeated by others in the world}}. Physics results (relations between
  observables) are independent of representations chosen for the mathematical objects
  needed, and independent of where and when (space and time chosen).  They should be
  repeatable by other physicists at other laboratories in the world. Therefore the actual
  spreading of the information on the findings to other laboratories in the world
  {\emph{is}} part of the operational procedure to gain physics insight.
\end{description}
To strengthen our intuition about the crucial step {\bf{DP3}}, let us consider an example:
Assume we want to measure the position in space of a physical object in classical
mechanics.  First we design a physically constructible `detector' covering a finite space
area $(x_i,\Delta x_i)$ which can distinguish whether the object is inside the detector
area or not.  Then we buy very many of these detectors and plaster (non-overlapping,
touching detectors) the physical space.  By checking all of them we learn in which
detector $(x_i ,\Delta x_i) $ the object is to the precision $\Delta x_i$. Repeating the
experiment but with (may be a more expensive) detector set with {\emph{finitely}} smaller
detector space, say $\Delta x_{i+1} = \Delta x_i/2$ will give a better precision of the
experiment. Repeating the application of the rule, which is obviously independent of the
absolute value of precision gained in a certain step, would give us the ideal physics
result. However we cannot experimentally do or pay for many refinement steps, and have to
fear that the correctness of the experiment will break down if we physically go too far.
That is why the limit process for the {\emph{physical observable}} is done by virtually,
not physically, going to the limit and mapping the result to a mathematical object as the
mathematical representative of the observable.  Each of the assumed algebraic properties
has then to be tested by respective physical experiments. Thus only after experimental
testing e. g. all commutative algebra properties of the mathematical representative of the
space position observable we can identify it with a vector in Euclidian space.
 
In short, we need the process given in the {\bf{DP}} steps to ensure the choice of the
related mathematical object and to get the best strategy for a semantic encoding of
physics in a markup language. But how does this formal {\emph{operational definition}} fit
to the actual practical fixing of physics observables by international committees, e.g.
the CGPM (Conference Generale des Poids and Measures), CODATA, IUPAP (International Union
of Pure and Applied Physics), and SUNAMCO (Standards, Units and Nomenclature, Atomic
Masses and Fundamental Constants)? This question is the domain of {\emph{Metrology}}, an
active research field of its own (see~\cite{metrology} for a recent summary).  The
international metrologic commissions dwell on the next step of fixing observables once the
operational definition has been set, focusing on
\begin{description}
\item[precise measurement procedures] extending the practical measurement of observables
  to very large and very small scales is achieved (for the length scale from cosmological
  to subatomic).
\item[determining physical constants] by finding quantitative natural laws which connect
  real observations and thus can be reformulated to define a {\emph{physical constant}}
  which is given by a physical process (such as the gravitational constant, the speed of
  light, etc.).  Examples are: the scale for the time is set to be the {\emph{second}}
  fixed as 9.192.631.770 periods of the hyperfine split light radiation of the atom
  {\emph{Cesium}}.  The {\emph{metre}} is the length of the path traveled by light in
  vacuum during a time interval of $1/299.792.458$ of a second, thus replacing the
  Ur-metre at Paris measured by a length ruler.
\item[hunting for higher precision] which is especially necessary when long time unique
  series of measurements of given observables have to be trusted such as in geophysics,
  astrophysics.
\end{description}

\subsection{Principal Objects for a Physics Markup Language}

Given the above, we have to model the following principal concepts in a content/context
markup language for physics.

\begin{description}
\item[Observables] As described, an observable is defined by the operational description
  of the defining apparatus, an iterative refinement rule, and properties such as
  dimension, scale, and attached algebraic object. The relations in which this observable
  occurs, etc. can be represented in {\omdoc}.
\item[Experiments] Physics is distinct from other sciences by strictly sticking to
  reproducible experiments' outcomes as the source of knowledge. Reproducible means: to be
  able to tell others about the experiment so that they can reproduce it. This is in
  contrast to other sciences such as meteorology, history, or biology, which have records
  (data recorded over time) as principal source.
\item[Apparatuses] Experimental measurements are done using apparatuses.  An apparatus
  $\cal{A}$ is defined by a detailed description on how to build it, so that others may
  redo the experiment.  Alternatively an apparatus can be fully described by {\emph{all}}
  its simultaneously measurable observables\footnote{We note that in physics the list of
    properties of an apparatus is either finite or countably infinite (in contrast to e.g.
    biological systems).  This assures a Hilbert space of states and real numbered values
    for the observables as the eigenvalues of the Hermitean Operator representing the
    Observable.  This restriction to at most countable infinite property list is
    absolutely essential for physics.  Only by that we get, after mapping to the formal
    mathematical context the correct observation that in all physics experiments measured
    numbers are real, as assured by the Hilbert state space and the Hermitean Operators
    therein. }.  A set of given values for all its properties defines a {\textbf{State}}
  $|a_i\rangle$, $i=1,2,3,\ldots$ of $\cal{A}$.  An experiment is conducted by bringing
  $\cal{A}$ into contact with another apparatus $\cal{B}$. The logical asymmetry of a
  typical experiment comes only with the mind of the observer, the experimentalist. She
  uses $\cal{B}$ to get information about the state of $\cal{A}$.
\end{description}

Again, an example is in order: Assume we have an apparatus {\emph{gas-filled bottle}},
with a set of observables such as density of {\emph{gas}}, {\emph{size}}, {\emph{color}},
and {\emph{material}}, and one thermostatic observable, the {\emph{temperature}}.  We
choose $\cal{B}$ to be a device which mostly has the same observables, such that if
brought into thermal contact with $\cal{A}$ it does not affect the properties of $\cal{A}$
significantly but adapts its temperature to that of $\cal{A}$.  We call $\cal{B}$ a
{\emph{thermometer}} and $\cal{A}$ a {\emph{system}} in this case.  Our interest here is
on the value shown by the thermometer as a result of the value of the temperature of the
apparatus.  We neglect the (inevitable) changes of other observable's values, both of the
system and the measurement device: such idealizations of experiments are common in
physics.
%%% Local Variables: 
%%% mode: stex
%%% TeX-master: "mkm06"
%%% End: 

% LocalWords:  Galilei Bridgeman spacepoint timeclick sizably mathema DP CGPM
% LocalWords:  Ehrenfest Poids CODATA IUPAP SUNAMCO Pendrill metre Ur Hermitean
% LocalWords:  stex mkm
