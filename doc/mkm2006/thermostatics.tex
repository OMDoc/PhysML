\section{Case Study: Thermostatics}

\begin{newpart}{moved here from the intro}
  In the following we will exemplify this approach by application to a small subfield of
physics, {\emph{thermostatics}}, which has some practical advantages to start with:
\begin{itemize}
\item it is non-trivial;
\item it is a linear theory;
\item the thermostatic state space is affine (no metric) and mostly has just very few
  dimensions,, --- with just exactly one thermal;
\item the variables have a range;
\item it is mathematically confined to the subfield of operations with partial
  derivatives\footnote{More generally: of total derivatives for an affine space.};
\item it is of the widest most interest in numerous technical applications, so that a
  semantic encoding will be of utmost utility.
\end{itemize}
\end{newpart}
The physics of thermostatics is extremely simple.  All physically measurable relations and
observables can be cast mathematically as total differentials (represented by sets of
partial derivatives) of non-observable {\emph{Potentials}}.  Any observable, as a function
of its specific variables, contains the whole physics information about the system.  The
nice thing is that no metrics are involved, id est the thermostatic space is affine.

So what do we have to do to be able to make use of a {\physml} to check the
correctness of presented equations: 
\begin{enumerate}
\item define the variables as principal objects.
\item define the measurable variables for a given experimental setup.
\item setup the total derivative with its partial derivatives.
\item encode the Jacobi determinant for coordinate transformations.
\item setup practical tools for checking the correctness of transformations for linearity.
\item Fortunately the dimension check as the easiest test of an operating {\physml} is simple in
  thermostatics.
\end{enumerate}

\subsection{The Formal Part of {\physml} Exemplified}
Now we exemplify the formal part of {\physml} in the nutshell of thermostatics:

All {\emph{physical variables}}, such as {\emph{temperature}} $T$, {\emph{entropy}} $S$,
{\emph{internal energy}} $U$, {\emph{density}}, {\emph{chemical potential}}, etc. are real
numbered variables (with the option to extend to the complex plane).

Each having its own {\emph{physics dimension}}, such as {\emph{energy}}, {\emph{density}},
etc.

Each having its own {\emph{range}}, such as $1/T < \infty$, or $\rho>0 $, etc.

Each variable is defined by an experimental setup able to measure it.

Given a thermostatic system and an experimental setup, this defines which variables can be
measured simultaneously, the number of these define the {\emph{dimension of thermostatic
    space}}.  Mostly this is two or three.

Given a full set of simultaneously measurable variables of a given system (say
$SV,N$),then the universal law of thermostatics says: {\emph{all}} physical information
attainable on this system under this and any other experimental setup with other
thermostatic variables given are contained in {\emph{one}} real valued function on the
given set of experimental variables.  This quantity itself is not measurable and thus
called {\emph{The Thermostatic Potential for the given setup}}. Its total differential,
represented by its partial derivatives then lead by the usual algebra of partial
derivatives under the respective Jacobi transformations to any physics quantity under any
other experimental setup.  That is, why thermostatics is so simple as an example of
{\physml}.

So we need:
\begin{itemize}
\item real numbered variables
\item range conditions
\item potentials and its representations
\item Jacobi algorithm for thermostatic space coordinate transformation
\item partial derivatives algebra
\item ordinary 12 class mathematics such as equation solving, derivative, stretching of
  one variable, inversion, addition and multiplication.
\end{itemize}

\subsection{Bausteine/Steinbruch}\label{sec:thermostatik}

Examples of {\emph{thermostatic observables}} are pressure $p$, Temperature $T$, Entropy
$S$, internal energy $U$.


A given experimentally prepared {\emph{thermostatic system}} is defined by the {\emph{set
    of all simultaneously measurable observables}} of that system.  Their number is either
countable infinite or much less, in most practical cases rather small. In any case the set
contains exactly {\emph{one}} thermostatic variable\footnote{That is the essence of the
  ``zeroth law of thermostatics'' or of the initiator Robert Mayer. The other laws of
  thermostatics are (1.: Energy conservation including the thermal one; 2. Entropy can
  only rise in a closed system; 3. temperature axis has a cut at $T=0$.}  All other
observables of a set are non-thermal, called 'mechanic', like {\emph{volume}},
{\emph{density}}, {\emph{mass}}, {\emph{charge}}, {\emph{angular momentum}},
{\emph{composition}},\ldots Thus the essence of thermostatics can be demonstrated with
simple systems which have just one non-thermal variable, such as the all-textbook-famous
{\emph{gas container}}: Inside a homogeneous gas of pressure $p$ and put into a piston for
compression and a heater for enforcing a homogeneous temperature $T$.

The range of the variables can be read off from the dictionary of thermostatic
observables.

The {\emph{values}} of the variables are determined by an experiment.  We know beforehand
that they are real numbered.

The {\emph{Theorem}} or Axiom or Law of thermostatics says, that there exists for a given
system and set of observables exactly {\emph{one}} function of the observables, called the
{\textbf{thermostatic potential}} for that given experimentally prepared system, from
which {\emph{all}} measuring results in all possible experimental setups of the same
system including such with any other set of observables chosen, can be calculated
algebraically from its total derivative.  However it is not said, which function.  The
{\emph{first law of thermostatics}} then gives us a clue how to construct it: the
{\emph{internal energy density $u$}} change is equal to the thermal energy change plus the
non-thermal energy change, and luckily $u$ is the thermostatic potential for the variables
entropy $s$ and density $\rho$:
\begin{equation}
{\rm d} u := T {\rm d}s + p/\rho^2 {\rm d}\rho \quad ;
\end{equation}
While $u$ is a non-measurable thermostatic potential, its total derivative\footnote{We
  mean the mathematical algebra of Cartan's exterior derivative.} is represented by its
independent first order partial derivatives, one for each variable.  By use of the
Jacobian (thermostatic space) coordinate transformations we can calculate the set of
partial derivatives of one potential to that of another and thus predict experiments under
different conditions than the one we started with.

An Example: the {\emph{specific heat}} $c_p(T) := (\partial f/\partial T)_p$ as a function
of temperature for fixed pressure, giving the free energy $f$ increase per degree heated,
may have been measured. However, now we want to know the specific heat $c_V(T) = (\partial
u / \partial T)_V$.  \footnote{The two potentials $f$ and $u$ differ by the mechanical
  work to expand the system during the measurement, in contrast to measuring for fixed
  volume.  $p {\rm d}v$, with $u + pv =: f$.  The free energy has $(T,p)$ as variables,
  while the internal energy has $(T.\rho)$.}  This simple calculation has first been done
to our knowledge by Max Planck in his fundamental textbook {\sl
  Thermodynamik}\cite{planck-1905} of 1905.\footnote{Henceforth we make use of a modern
  notation of partial derivatives\cite{hilf-suessmann-1972}, first introduced by
  A. Buchler, which separates in the notation the subject, the mapping and the result of
  the derivative, $(\partial a/\partial b)$ for constant $c =: \partial_b^c a$.}.  Planck
(on his page 55) gets from ${\rm d}u := T {\rm d}s + p {\rm d}v$, also $T =
\partial_s^v u $ and $ p = \partial_v^s u$ the useful relations
(remember: $u$ is a potential for the variables $(s,v)$ not for $(T,\rho)$)
\[\partial_p^v \, u = c_v \, \partial_p^v T \qquad \partial_v^p u = c_p \, \partial_v^p T - p
\quad ,\]
and from this  finally
\[(c_p - c_v)\,  \partial_v^p\partial_p^v T +  \partial_p^v c_p \, \partial_v^p T
-  \partial_v^p c_v  \, \partial_p^v T  = 1 \quad .\]
This nice identity between specific heats for different experimental setups
is technically used for cross-checking experimental numbers.

It is the aim of this paper to enable in the near future automatic checking tools
triggered from a digital thermostatics paper, which had been semantically encoded
with {\physml} to check even such nontrivial identities.

For this we have to stratify the algebra of partial derivatives for automated
semantic encoding using {\omdoc}.


We start from the {\emph{total Differential}} ${\rm d} := {\rm d}x \partial_x^y + {\rm
  d}y \partial_y^x $ and its behavior under affine Coordinate Transformations in the
thermostatic space $(x,y) --> (a,b)$.  We get the general rules
\[\partial_x^y = \partial_x^y a \quad \partial_a^b + 
\partial_x^y b \quad \partial_b^a \quad , \]
with its simple special cases
\[\partial_x^y = \partial_x^y z \, \partial_z^y \quad {\rm chain rule},
\partial_x^y z \quad \partial_z^y x = 1 \quad {\rm inversion}, \quad
\partial_x^y z  \,  \partial_z^x y  \, \partial_y^z x = - 1 \quad {\rm triangulation}. \]

\paragraph{What tools do we expect}
Mathematical objects in equations of a paper in physics are identified to
be the algebraic image of its respective physical observable.

Physical dimensions, out of range of variables, units could be tested by automatic tools.

Partial derivatives containing equations could be checked for its validity
using one of the algebraic tools able to make use of Jacobian transformations.

Making use of a future {\physml} also allows each author to choose his own symbols
for a given observable, allows to centrally store and thus link to the
accumulated knowledge of this observable, to search for papers where these
quantities occur independent of the actual representation or notation.

Specifically in thermostatics, once semantic encoding with {\physml} is  achieved,
making use of the axioms of thermostatic, and the algebraic derivations as said,
one  could automatically calculate predictions for other experimental setups
than the one used to get the necessary full thermostatic information.

Technically, an authoring tool will be developed which enables the author to easily make
the necessary semantic markup by tags within his/her {\LaTeX}-typing.

%%% Local Variables: 
%%% mode: LaTeX
%%% TeX-master: "mkm06"
%%% End: 

% LocalWords:  est Jacobi Bausteine Steinbruch Thermodynamik Buchler mkm
