\section{Reading, Writing and Arithmetic with {\physml} Documents}\label{sec:physml-stex}

Of course, the XML-based {\physml} format presented here is not directly suited for humans
to read and write. And indeed it is not intended to be; humans should use adaptive
presentations for reading and invasive editors~\cite{KohKoh:cdad04} for manipulating
{\physml} documents.

The {\omdoc} style sheets have been extended appropriately for the {\physml}-specific
elements. With these, {\physml} documents can be converted to XHTML documents with
{\mathml} formulae that can be displayed in a browser or to PDF documents for printing via
the {\LaTeX} formatter.

{\physml} inherits a well-established notation declaration language and presentation
system from the {\omdoc} format: for new concepts that are introduced via
{\element{symbol}} elements notation information can be specified via {\omdoc}
{\element{presentation}} elements: In the presence of the following declaration,
\begin{lstlisting}
<presentation for="#Celsius">
  <use format="html|pmml">&#x000B0;C</use>
  <use format="TeX">{}^{\circ}C</use>
</presentation>
\end{lstlisting}
The {\openmath} object representing the temperature in of the thermometer in
{\mylstref{experiment}} will indeed look like the visualization in the box. 

To write {\physml} documents, we have concentrated on the {\LaTeX} workflow that is
well-established in physics. Concretely, we have extended the semantic {\TeX} system
{\stex}~\cite{Kohlhase:albwo06} by {\physml} functionality. 

\setbox0=\hbox{\footnotesize\stex}
\begin{lstlisting}[label=lst:units-stex,caption=Writing the {\physml} for
  {\mylstref{units}} in \usebox0] 
\begin{module}[id=units,uses=dimensions]
  \symdef[type=$\mass$]{gram}{g}
  \symdef[type=$\temperature$]{Kelvin}{K}
  \symdef[type=$\temperature$]{Celsius}{{}^{\circ}C}
  \begin{definition}[for=Celsius]
    $\allcdot{x>0}{x\Kelvin=(x-273.15)\Celsius}$
  \end{definition}
\end{module}
\end{lstlisting}

For more choice in invasive editors, we will extend the {\omdoc} wiki
system~\cite{LanKoh:swmkm06} and the PowerPoint plugin for {\omdoc}~\cite{KohKoh:cdad04}
to {\physml}.

The explicit, and standardized content representations for physical documents in {\physml}
will allow us to offer added-value services that cannot be offered on conventional
representations.  Examples are the dimension check comparing the physical dimensions, and
the units used in an equation presented in a paper. If the dimensions on both sides of an
equation do not match (say {\emph{kg}} on one side, and {\emph{meter}} on the other, the
equation is physically openly wrong, if different units for the same dimensions were used
on both sides this is called `unlawful sloppiness' (say $\Kelvin$ on one side, $\Celsius$
on the other).  Other checks will include the algebraic matching of both sides of an
equation (say if {\emph{vector}} on one side and {\emph{coaxial vector}} on the other,
this equation is bluntly incorrect). But more intelligent codes could also read the
semantics delivered and offer mapping of algebraic results in different representation
(say: integral instead of differential formulation, vector vs.  vector-component or
exterior form, etc.) thus directly assisting the reader to not having to read clumsy
formulations of theoretical results from old times, but get it in the present used
representations and notations.
%%% Local Variables: 
%%% mode: stex
%%% TeX-master: "mkm06"
%%% End: 

% LocalWords:  stex pmml lst wiki plugin mkm
