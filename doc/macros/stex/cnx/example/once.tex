\documentclass[a4paper]{cnx}
\usepackage{cmathml,cmathmlx}
\begin{document}

\begin{cnxmodule}[id=m0001,name=Session 1: Set theory in the science of complex systems.]

\begin{metadata}[version=1.0,created=2006/08/10]
  \begin{authorlist}
    \cnxauthor[id=jj,firstname=Jeffrey,surname=Johnson,email=j.h.johnson@open.ac.uk]
  \end{authorlist}
  \begin{maintainerlist}
    \maintainer[id=ka,firstname=Katarina,surname=Alexiou,email=a.alexiou@ucl.ac.uk]
    \maintainer[id=tz,firstname=Theodore,surname=Zamenopoulos,email=t.zamenopoulos@ucl.ac.uk]
  \end{maintainerlist}
  \begin{keywordlist}
    \keyword{Sets}
    \keyword{Logic}
    \keyword{Formulation of theoffries of complex systems}
  \end{keywordlist}
  \begin{cnxabstract}
    dummy abstract
  \end{cnxabstract}

\end{metadata}

%% TODO
%% o Expand title information --> session1-pure.tex
%% o Generate table of contents --> session1-pure.tex
%% o Meta-data:
%%   o remove last comma
%% o Content MathML
%%   o eqnarray, cequation?
%% o What about footnotes? Shall they be translated to notes?
%% o Item separation in itemize environment too wide
%% o footnote of cquote too close
%% o If a cexercise is followed by a cpara, then an empty line has to be put in front of
%%   cpara. Otherwise the layout is messed up. 
%% o Symbol for a universal set.
%% o Symbol for a symmetric difference
%% o Layout of \Capply does not work
%% o Symbol for a tuple
%% o Translation of `cfigure' does work yet

\begin{ccontent}
\begin{csection}[id=intro]{Get going}
  \begin{cpara}
    Sets are fundamental in mathematics. They are the basic building blocks of just
    about everything. Sets are intimately tied up with logic, and therefore their use
    has a great deal to offer in the formulation of theories of complex
    systems. Fortunately, the basic ideas of set theory are very simple and easy to
    understand, so let's get going.
  \end{cpara}
\end{csection}

\begin{csection}[id=sets-and-elements]{Sets and elements}
  \begin{cpara}
    Any well-defined collection of objects forms a {\term{set}}. The objects are called
    elements or members of the set. Examples of sets include:
  \end{cpara}
  \begin{cexample}[id=sets-and-elements-ex]
    \begin{itemize}
    \item the British monarchs since the Norman conquest
    \item the even numbers
    \item people who work for the Treasury
    \item the planets
    \item cells in a body
    \item the letters of the alphabet
    \item the students studying this course
    \end{itemize}
  \end{cexample}
  \begin{cpara}
    Members of a set may themselves be sets. Sets of sets are often called
    {\term{classes}}. For example, one could define a set of sets of people classified by
    their jobs, ages, incomes, or health.
  \end{cpara}
  \begin{cpara}
    The notion of being {\term{well defined}} lies at the heart of mathematics and, I
    think, science. It will be defined more precisely later in the course. For the moment
    let us say that a set $X$ is well defined when for any element x we can decide whether
    it is a member of $X$ or not. If there exists such a procedure for deciding
    membership, then we will say that the set is {\term{grounded}} or {\term{effectively
        decidable}}.
  \end{cpara}
  \begin{cpara}
    As we shall see, this definition needs refining for multi-valued and time-dependent
    logics. For example, Prince William does not belong to the set of British Kings, but
    he may belong to it at some future date. Most sets have dynamic membership, the
    exceptions above being the even numbers and the letters of the alphabet. We can argue
    whether or not the set of planets is fixed. Certainly the set of {\term{known}}
    planets has changed through time. Whether or not this belongs to a {\term{fixed}} set
    of planets with members that we don't yet know about is a matter for
    speculation. Indeed, we can hypothesise that there is a {\term{universal set}} of
    planets that contains all the planets that we know and all the others yet to be
    discovered. Then the set of known planets is a `subset' of the universal set of
    planets.
  \end{cpara}
  \begin{cpara}
    SAQs are self-assessed questions, intended to help you learn. The answers are given at
    the end of the document.\medskip
  \end{cpara}
  \begin{cexercise}[id=saq1,name=SAQ]
    \begin{cproblem}[id=saq1p]
      \begin{enumerate}
      \item Is the set of even numbers grounded?
      \item Is the set of planets grounded?
      \item Is the set of poor people grounded?
      \end{enumerate}
    \end{cproblem}
    \begin{csolution}[id=saq1s]
      Your answers may differ to mine:
      \begin{enumerate}
      \item A procedure for seeing if a number is even is to see if it its last digit is
        0, 2, 4, 6, or 8, so the set is grounded. 
      \item There are well established procedures for recognising the known planets, so
        this set is grounded.
      \item There are definitions for people being poor in terms of them being below given
        income thresholds, and in this respect the set is grounded. As the term is used in
        many conversations it is not grounded.
      \end{enumerate}
    \end{csolution}
  \end{cexercise}
\end{csection}

\begin{csection}[id=intro-math-notation]{An introduction to mathematical notation}
  \begin{cpara}
    Mathematicians make extensive use of specialised notation --- not as a secret language
    to hide the obvious, but because using a symbol like $\Ceq{}$ is much less tedious
    than writing out the word {\term{equals}} when used many time. Many people are
    surprised to learn that mathematical notation is a relatively recent thing. For
    example, in the passage below, Fermat is using ``aeq'' because the modern equals sign,
    $\Ceq{}$, had not yet been invented\footnote{According to Wikipedia,
      ({\link[src=http://en.wikipedia.org/wiki/Robert_Recorde]{Robert Recorde}},
      referenced 14$^{th}$ August 2006), the Welshman, Robert Recorde, introduced the
      equals sign, $\Ceq{}$, in 1557.}:
    \begin{cquote}
      ``It was not until the second half of the 16$^{th}$ century that developments occurred
      in the use of symbolic notations that could really merit description as the beginnings
      of algebra. Fran\c cois Vi\`ete (1540-1603) was among the first clearly to be aware of
      the significance of symbols that generalise numbers or magnitudes. He introduced the
      use of a vowel for a quantity assumed to be unknown or undetermined and consonants for
      quantities that are assumed to be given. The idea of `given' quantities being
      represented by a letter is the beginning of the role of modern variables as
      `placeholders' for values. Thus Boyer remarks ``Here we find for the first time in
      algebra a clear-cut distinction between the important concept of a parameter and the
      idea of unknown quantity''\ccite{Boyer, C., A history of mathematics, Wiley, 1968.}
    \end{cquote}
  \end{cpara}
  \begin{cpara}
    In 1636 Fermat (who has studied Vi\`ete) was writing equations like \[A\; in\; E\;
    aeq\; Z\; pl.\] The Latin `in' means `times' and the `pl' is an abbreviation for
    `planus`\footnote{`planus' means plane or surface.}. He showed that the corresponding
    locus is a hyperbola (c.f. $\Ceq{\Ctimes{x,y},\Cpower{c}{2}}$).\footnote{Fermat, P.,
      {\ccite{Isagoge ad locos planos et solidos}}, 1679. [written in 1636]} Vi\`ete and
    Fermat (at this time) still regarded the magnitudes associated with letters in a
    geometrical way so that a product was an area etc.''\footnote{Beynon, M, Russ, S,
      `Variables in mathematics and computer science', in {\ccite{The mathematical
          revolution inspired by computing}}, J. Johnson and M. Loomes (eds), Claredon
      Press (Oxford), 1991.}
  \end{cpara}
  \begin{cpara}
    Mathematical notation is {\term{created}}, and it can be more or less good for its
    purpose. In the introduction to Wittgenstein's {\ccite{Tractatus
        Logico-Philosophicus}}, Bertrand Russel writes
    \begin{cquote}
      ``$\dots$ a good notation has a subtlety and suggestiveness which make it seem, at
      times, like a live teacher. Notational irregularities are often the first sign of
      philosophical errors, and a perfect notation would be a substitute for thought.''.
    \end{cquote}
    Not only is mathematical notation created, but the people who create often it invest a
    considerable amount of ego in it. Murray Gell-Man writes that
    \begin{cquote}
      ``a scientist would rather use another person's toothbrush than another scientist's
      nomenclature''.\ccite{Gell-Mann, M., `Plectics: `The study of simplicity and
        complexity', Europhysics News Vol. 33 No. 1, 2002,
        {\link[src=http://www.europhysicsnews.com/full/13/article5/article5.html]{Article5}}
        (referenced 14$^{th}$ August 2006).}
    \end{cquote}
  \end{cpara}
  \begin{cpara}
    When you have finished this course you will be in a position to invent your own
    notation to represent the complex systems that interest you. If your system has
    properties not possessed by any other system you will have to invent new notation,
    possibly with the help of a mathematician. You may do this anyway, and later discover
    a system that shares properties with yours developed using a different notation. Then
    you too will have to decide on whether you will change toothbrushes for the greater
    good of science, or if you will obscure the interdisciplinary connections and doggedly
    stick to your way of doing it. It's a tough choice. Fortunately before you face such
    dilemmas, a lot of mathematical notation is `standard', and easy to learn.\medskip
  \end{cpara}
  \begin{cexercise}[id=saq2,name=SAQ]
    \begin{cproblem}[id=saq2p]
      \begin{enumerate}
      \item Translate the following formula into words:
        $\Cneq{\Cplus{\Ctimes{\Ccn{17},\Ccn{49}},
                      \Cdivide{\Ccn{4}}{\Ccn{2}}}}
              {\Cdivide{\Ctimes{\Cminus{\Ccn{19}}{\Ccn{13}},
                                \Cplus{\Ccn{2},\Ccn{6}}}}
                       {\Ccn{4}}}$
      \end{enumerate}
    \end{cproblem}
    \begin{csolution}[id=saq2s]
      $\Cneq{\Cplus{\Ctimes{\Ccn{17},\Ccn{49}},
                     \Cdivide{\Ccn{4}}{\Ccn{2}}}}
            {\Cdivide{\Ctimes{\Cminus{\Ccn{19}}{\Ccn{13}},
                              \Cplus{\Ccn{2},\Ccn{6}}}}
                     {\Ccn{4}}}$
      translates as seventeen times forty nine plus four divided by 2 is not equal to
      thirteen subtracted from nineteen times one quarter of two added to six. The point
      of this SAQ is that you already know how to `speak' arithmetic because you know what
      the symbols mean. In this session you will encounter new symbols, but translating
      them into words is done in the same way.
    \end{csolution}
  \end{cexercise}
\end{csection}

\begin{csection}[id=basic-concepts-notation]{The basic concepts and notation of set theory}
  \begin{cpara}
    Usually sets are denoted by capital letters such as $A$, $B$, $C$, $\dots$, $X$, $Y$, $Z$,
    and their elements are denoted by lower case letters such as $a$, $b$, $c$, $\dots$, $x$,
    $y$, $z$.
  \end{cpara}
  \begin{cpara}
    There are two ways to specify sets. The first is called set definition by
    {\term{extension}} and involves listing the elements of the set in braces, e.g.:
    \begin{eqnarray*}
      &&\Cset{a, e, i, o, u} \\
      &&\Cset{rook, knight, bishop, king, queen, pawn} \\
      &&\Cset{London, Paris, Berlin, Rome, Madrid, Budapest, \dots}
    \end{eqnarray*}
    The second way involves stating the properties which characterise the elements of the
    set. This is called set definition by {\term{intension}}, for example
    \begin{eqnarray*}
      V & = & \CsetRes{x}{x\; \mbox{is a vowel}} \\
      C & = & \CsetRes{x}{x\; \mbox{is a cheese piece}} \\
      E & = & \CsetRes{x}{x\; \mbox{is a European capital city}}
    \end{eqnarray*}
    We write $\Cin{x}{X}$ to mean ``x is an element of the set X'' or, equivalently, ``x
    belongs to X''. We write $\Cnotin{x}{X}$ to mean that ``x does not belong to X''. For
    example, we can write
    \begin{eqnarray*}
      &&\Cin{\mbox{Helsinki}}{E} \\
      &&\Cnotin{\mbox{Manchester}}{E}
    \end{eqnarray*}
    Two sets are {\term{equal}} when they have exactly the same elements. Formally, $\Ceq{A,B}$,
    if and only if $\Cin{a}{A}$ implies $\Cin{a}{B}$, and $\Cin{b}{B}$ implies
    $\Cin{b}{A}$.
  \end{cpara}
  \begin{cpara}
    If $\Ceq{A,B}$ then $\Ceq{B,A}$.
  \end{cpara}
  \begin{cpara}
    In set theory, all the sets under consideration are assumed to be contained in some
    large fixed set called the {\term{universal set}} or {\term{universe}}, denoted $U$.
  \end{cpara}
  \begin{cpara}
    Above we suggested a universal set of planets, which is a useful idea for those who
    want to analyze planetary systems. A sociologist might consider a universal set of
    people, and an economist might consider a universal set of goods in a particular
    market. A complex systems scientist might use all these sets as sub-universes of a
    larger universe.
  \end{cpara}
  \begin{cpara}
    The set with no elements is called the {\term{empty set}}, or {\term{null set}}, which
    is denoted by the symbol, $\Cemptyset$, which is like a zero with a line through it.
  \end{cpara}
  \begin{cpara}
    For example, the set
    $\Ceq{\CsetRes{x}{x\; \mbox{is a pig and}\; x\; \mbox{can fly}},
          \Cemptyset}$.
  \end{cpara}
  \begin{cpara}
    A set can be empty with respect to a particular universe. In the previous example, $x$
    was restricted to the universe of pigs.
  \end{cpara}
  \begin{cpara}
    The empty set is {\term{unique}}, meaning there is only one empty set. If $A$ and $B$
    were both empty sets, then they have exactly the same elements (none), so
    $\Ceq{A,B}$.\medskip
  \end{cpara}
  \begin{cexercise}[id=saq3,name=SAQ]
    \begin{cproblem}[id=sqq3p]
      Read the follwing in words:
      \begin{enumerate}
      \item Let $\Cin{p}{\CsetRes{x}{x\;\mbox{is a cheese piece}}}$ and
        $\Cin{p}{\CsetRes{x}{x\; \mbox{white}}}$.
        Then $\Cnotin{p}{\CsetRes{x}{x\;\mbox{is a black cheese piece}}}$ 
      \item Is it true that
        $\Ceq{\CsetRes{x}{x\;\mbox{is a Dutch city and has a cricket team}},
              \Cemptyset}$?  
      \end{enumerate}
    \end{cproblem}
    \begin{csolution}[id=saq3s]
      \begin{enumerate}
      \item Let $\Cin{p}{\CsetRes{x}{x~\mbox{is a chess piece}}}$ and
        $\Cin{p}{\CsetRes{x}{x\;\mbox{white}}}$. Then $\Cnotin{p}{\CsetRes{x}{x~\mbox{is
              a black chess piece}}}$ This read as ``Let p belong to the set of chess
        pieces and let p belong to the set of white elements. Then p does not belong to
        the set of black chess pieces.''
      \item Is it true that $\Ceq{\CsetRes{x}{x\;\mbox{is a Dutch city and has a cricket
          team}},\emptyset}$? This reads as ``Is it true that the set of Dutch cities
        with a cricket team is empty?''. 
      \end{enumerate}
    \end{csolution}
  \end{cexercise}

  \begin{cpara}
    A set $A$ is a {\term{subset}} of the set $B$ if every element of $A$ is an element of
    $B$. This is written as $\Csubset{A,B}$ or, equivalently, $\Csupset{B,A}$.
  \end{cpara}
  \begin{cpara}
    $A$ is a {\term{proper}} subset of $B$ if $B$ contains at least one element not
    possessed by $A$. This is written as $\Cprsubset{A,B}$ or, equivalently, $\Cprsubset{B,A}$.
  \end{cpara}
  \begin{cpara}
    For example, the set of women is a proper subset of people. The set of economically
    active people is a proper subset of the set of people in Britain.\medskip
  \end{cpara}
  \begin{cexercise}[id=saq4,name=SAQ]
    \begin{cproblem}[id=saq4p]
      Read the follwing in words:
      \begin{enumerate}
      \item Is it true that if $\Csubset{A,B}$ and $\Csupset{B,A}$ then $\Ceq{A,B}$?
      \end{enumerate}
    \end{cproblem}
    \begin{csolution}[id=saq4s]
      Is it true that if $\Csubset{A,B}$ and $\Csupset{B,A}$ then $\Ceq{A,B}$ translates
      as ``Is it true that if A is a proper subset of B and B is a proper subset of A then
      A equals B?''. It's not! It's not possible for $A$ to be a proper subset of $B$ and
      for $B$ to be a proper subset of $A$.
    \end{csolution}
  \end{cexercise}

  \begin{cpara}
    The {\term{intersection}} of two sets $A$ and $B$, written $\Cintersect{A,B}$ is the
    set of elements that belong to $A$ and belong to
    $B$: \[\Ceq{\Cintersect{A,B},\CsetRes{x}{\Cin{x}{A}\; \mbox{and}\; \Cin{x}{B}}}\] For
    example, the intersection between the set of people living in poverty and the set of
    people with bad health is the set of people living in poverty with bad health.
  \end{cpara}
  \begin{cpara}
    The intersection of the set of over eighty years old, and the set of premier division
    footballer is (to my knowledge) empty: which could be written
    $\Ceq{\Cintersect{\mbox{eighty\_year\_olds},\mbox{premier\_division\_footballers}},\Cemptyset}$.
  \end{cpara}
  \begin{cpara}
    When two sets have empty intersection, i.e. they have no elements in common, they are
    said to be {\term{disjoint}}.
  \end{cpara}
  \begin{cpara}
    The {\term{union}} of two sets $A$ and $B$, written $\Cunion{A,B}$ is the set of
    elements that belong to either of them: \[\Ceq{\Cunion{A,B}, \CsetRes{x}{\Cin{x}{A}\;
        \mbox{or}\; \Cin{x}{B}}}\] For example, the union of the adults of people who are
    men and the set of adults who are women is the set of people who are men or women.
  \end{cpara}
  \begin{cpara}
    Let $\mathbf{U}$ be a universal set, and let $A$ be a subset of $\mathbf{U}$. The
    complement of $A$ in $\mathbf{U}$ is defined to be the set
    $\Ceq{\Ccomplement{A},\CsetRes{x}{\Cin{x}{\mathbf{U}}\; \mbox{and}\; \Cnotin{x}{A}}}$,
    the set of elements that belong to $\mathbf{U}$ but do not belong to $A$.
  \end{cpara}
  \begin{cpara}
    Intersection and union are said to {\term{operate}} on sets, and they are often called
    set {\term{operators}} or set operations.\medskip
  \end{cpara}
  \begin{cexercise}[id=saq5,name=SAQ]
    \begin{cproblem}[id=saq5p]
      Read the follwing in words:
      \begin{enumerate}
      \item $\Ceq{\Cintersect{A,A},A}$
      \item $\Ceq{\Cintersect{\Cintersect{A,B},C},\Cintersect{A,\Cintersect{B,C}}}$
      \item $\Ceq{\Ccomplement{A}, \CsetRes{x}{\Cin{x}{\mathbf{U}}\; \mbox{and}\; \Cnotin{x}{A}}}$
      \end{enumerate}
    \end{cproblem}
    \begin{csolution}[id=saq5s]
      \begin{enumerate}
      \item $\Ceq{\Cintersect{A,A},A}$ reads as A intersection A equals A
      \item $\Ceq{\Cintersect{\Cintersect{A,B},C},\Cintersect{A,\Cintersect{B,C}}}$ reads
        as A intersection B intersected with C equals A intersected with B intersection C. 
      \item $\Ceq{\Ccomplement{A}, \CsetRes{x}{\Cin{x}{\mathbf{U}}\; \mbox{and}\;
            \Cnotin{x}{A}}}$ reads as A complement is the set with members that belong to
        the universal set but do not belong to A.
      \end{enumerate}
    \end{csolution}
  \end{cexercise}
\end{csection}

\begin{csection}[id=repres-sets-ops-diagrams]{Representing sets and set operations by
    diagrams}
  \begin{cpara}
    \cfigure[id=basicsets,%
           type=image/pdf,%
           caption={Illustrating basic set relationships and operations using diagrams}]%
           {width=.9\textwidth}{img/basicsets}%
   \end{cpara}
   \begin{cpara}
    Sets are commonly represented by circles and other enclosed areas. For example Figure
    {\ref{basicsets}}(a) shows $A$ as a subset of $B$. Figure {\ref{basicsets}}(b) shows two
    sets $A$ and $B$ with their intersection, Figure {\ref{basicsets}}(c) shows the union of
    $A$ and $B$, while Figure {\ref{basicsets}}(d) illustrates two disjoint sets.
  \end{cpara}
  \begin{cpara}
    Figures {\ref{setdiagrams}}(a), (b) and (c) show the universal set as a rectangle with
    other sets in it. This enables the complements of sets to be drawn as the shaded areas
    shown.
    \cfigure[id=setdiagrams,%
             type=image/pdf,%
             caption={Representing sets, their intersection, unions and complements by diagrams}]%
             {width=.9\textwidth}{img/setdiagrams}% 
    Diagrams like those shown in Figures {\ref{basicsets}} and {\ref{setdiagrams}} are
    called {\term{Venn diagrams}} after the logician John Venn (1834 - 1923). Charles Dodgson
    (Lewis Carol) is said to have introduced the notion of rectangle for the universal
    set\footnote{\link[src=http://www.lewiscarroll.org/religion/venn.html]{Venn}}. In fact the
    use of circles to represent sets began well before the nineteenth century. Some authors
    refer to {\term{Euler circles}}, and {\term{Euler diagrams}} which differ from Venn diagrams by
    allowing some of the sets to be
    disjoint\footnote{\link[src=http://en.wikipedia.org/wiki/Euler_diagram]{Euler diagram}}.\medskip 
  \end{cpara}
  \begin{cexercise}[id=saq6,name=SAQ]
    \begin{cproblem}[id=saq6p]
      \begin{enumerate}
      \item draw a diagram to illustrate $\Cprsubset{A,B}$.
      \item draw a diagram with sets $A$, $B$ and $C$ and shade the set $\Cunion{\Cintersect{A,B},C}$
      \item draw a diagram sets $A$, $B$ and $C$ and shade the set $\Cintersect{A,\Cunion{B,C}}$.
      \item What can you conclude from (b) and (c)
      \end{enumerate}
    \end{cproblem}
    \begin{csolution}[id=saq6s]
      \begin{enumerate}
      \item draw a diagram to illustrate $\Cprsubset{A,B}$.
        \cfigure[id=fig1,type=image/pdf]{width=.3\textwidth}{img/saq6a}
      \item draw a diagram with sets $A$, $B$ and $C$ and shade the set
        $\Cunion{\Cintersect{A,B},C}$
        \cfigure[id=fig2,type=image/pdf]{width=.3\textwidth}{img/saq6b}
      \item draw a diagram sets $A$, $B$ and $C$ and shade the set
        $\Cintersect{A,\Cunion{B,C}}$.
        \cfigure[id=fig3,type=image/pdf]{width=.3\textwidth}{img/saq6c}
      \item What can you conclude from (b) and (c)? From these two diagrams I can conclude
        that $\Cneq{\Cneq{\Cunion{\Cintersect{A,B}}}{C}}{\Cintersect{A,\Cunion{B,C}}}$
      \end{enumerate}
      \end{csolution}
  \end{cexercise}

  \begin{cpara}
    The {\term{difference}} of two sets $A$ and $B$, written $\Csetdiff{A}{B}$ is defined to
    be the set whose elements belong to $A$ but not $B$, $\Ceq{\Csetdiff{A}{B},\CsetRes{x}{\Cin{x}{A}\; \mbox{and}\; \Cnotin{x}{B}}}$. This can be illustrated as shown in
    Figure {\ref{diffsymset}}(a).
  \end{cpara}
  \begin{cpara}
    The {\term{symmetric difference}} (Figure {\ref{diffsymset}}(b)) of two sets is the set
    of elements that belong to $A$ or $B$ but not both. It is written as\footnote{A number of
      notations are used for symmetric difference:
      \link[src=http://mathworld.wolfram.com/SymmetricDifference.html]{Symmetric difference}}
    $A \Delta B$, where $\Delta$ is the Greek symbol `capital delta'.  
    \cfigure[id=diffsymset,%
            type=image/pdf,%
            caption={Diagrams shows set difference and symmetric difference.}]%
            {width=.9\textwidth}{img/diffsymset}% 
  \end{cpara}
  \begin{cexercise}[id=saq7,name=SAQ]
    \begin{cproblem}[id=saq7p]
      \begin{enumerate}
      \item Draw a diagram to show that $\Ceq{\Cunion{\Csetdiff{A}{B},\Csetdiff{B}{A}},A
          \Delta B}$ 
      \item Draw a diagram to show that $\Ceq{\Cintersect{A \Delta B,C},\Cintersect{A,C}
          \Delta \Cintersect{B,C}}$
      \end{enumerate}
    \end{cproblem}
    \begin{csolution}[id=saq7s]
      \begin{enumerate}
      \item Draw a diagram to show that $\Ceq{\Cunion{\Csetdiff{A}{B},\Csetdiff{B}{A}},A
          \Delta B}$ \cfigure[id=fig4,type=image/pdf]{width=.9\textwidth}{img/saq7a}
      \item Draw a diagram to show that $\Ceq{\Cintersect{A \Delta B,C},\Cintersect{A,C}
          \Delta \Cintersect{B,C}}$
        \cfigure[id=fig5,type=image/pdf]{width=.9\textwidth}{img/saq7b}
      \end{enumerate}
      \end{csolution}
  \end{cexercise}

  \begin{cpara}
    Euler and Venn diagrams are very useful for illustrating and exploring the intersections
    and unions of sets, but the pictures do not represent rigorous proofs.
  \end{cpara}
\end{csection}

\begin{csection}[id=subscripts-indexed-sets]{Subscripts and indexed sets}
\begin{cpara}
  In mathematics there are usually so many things that we run out of letters to represent
  them all. This is one of the reasons that Greek letters are used in mathematics --- it's
  because all the roman letters have been used for something else. Even so, there are not
  enough characters, so another trick is used, namely subscripting.
\end{cpara}
\begin{cpara}
    Let $X$ be a finite set with $n$ elements. Then let the elements of $X$ be numbered from $1$ to $n$,
    and write the $i^{th}$ numbered element be represented by the symbol $\Cselector{x}{i}$. Thus we can write
    \[\Ceq{X,\CsetRes{\Cselector{x}{i}}{\Ceq{i,{1,\dots,n}}},\Cset{ \Cselector{x}{1},
        \Cselector{x}{2}, \Cselector{x}{3}, \dots, \Cselector{x}{n}}}\] 
    which allows the elements $X$ to be represented by just one character, $x$ with a
    {\term{subscript}}, or small number or symbol on its it's lower right side. 
  \end{cpara}
  \begin{cpara}
    Apart from saving characters this has other advantages because the subscripted can be
    indexed using an `index set'.
  \end{cpara}
  \begin{cpara}
    Let $I$ be a set of numbers, $\Ceq{I,\Cset{1, 2, 3, 4, 5, \dots, n}}$. Then we can
    write $\Ceq{X,\CsetRes{\Cselector{x}{i}}{\Cin{i}{I}}}$. I is said to be an
    {\term{index}} set for the elements of $X$.
  \end{cpara}
\end{csection}

\begin{csection}[id=classes-sets-power-set]{Classes of sets and the power set}
\begin{cpara}
  In complex systems there are many heterogeneous sets. Sets can be arranged in {\term{classes}},
  where a class is a set of sets, and there can be classes of classes of sets. Classes of
  sets are often represented using a script font, e.g. below $C$ is a class of
  sets indexed by the set $\Cselector{I}{C}$ . In this example $I$ have added a
  subscript to the index set to link it explicitly to the class it indexes. 
  \[\Ceq{C,\CsetRes{\Cselector{C}{i}}{\Cin{i}{\Cselector{I}{C}}}}\] One
  of the most important classes of set is the {\term{power set}} of a $X$, which is the
  set of all subsets of $X$. The power set of $X$ is often denoted by
  $\Capply{P}X$. If a set has n elements, it can be shown that its power set
  has $\Cpower{2}{n}$ members, so the power set of $X$ is sometimes represented by the
  symbol $\Cpower{2}{X}$.
\end{cpara}
\begin{cexercise}[id=saq8,name=SAQ]
    \begin{cproblem}[id=saq8p]
      \begin{enumerate}
      \item What is the power set of the set $\Cset{a, b, c, d}$?
      \item How many elements are there in the power set of the set $\Cset{a, b, c, d}$?
      \item How many elements are there in the power of $\Cset{\Ccn{1}, \Ccn{2}, \Ccn{3},
          \Ccn{4}, \Ccn{5}}$? 
      \end{enumerate}
    \end{cproblem}
    \begin{csolution}[id=saq8s]
      \begin{enumerate}
      \item What is the power set of the set $\Cset{a, b, c, d}$?
        \begin{eqnarray*}
          \Capply{P}{\Cset{a, b, c, d}} & = & \Cset{ \Cemptyset, \Cset{a},
            \Cset{b}, \Cset{c}, \Cset{d}, \Cset{a,b}, \Cset{a,c}, \Cset{a,d}, \Cset{b,c},
            \Cset{b,d}, \Cset{c,d}, \Cset{a, b, c}, \Cset{a, b, d}, \Cset{a, c, d},
            \Cset{b, c, d}, \Cset{a, b, c, d}}
        \end{eqnarray*}
      \item How many elements are there in the power set of the set $\Cset{a, b, c, d}$?
        Including the empty set there are $\Ceq{\Cpower{\Ccn{2}}{\Ccn{4}},\Ccn{16}}$ sets in
        $\Capply{P}{\Cset{a, b, c, d}}$.
      \item How many elements are there in the power sets of $\Cset{\Ccn{1}, \Ccn{2}, \Ccn{3}, \Ccn{4}, \Ccn{5}}$?
        Including the empty set there are $2^5 = 32$ sets in $P(\{ 1, 2, 3, 4,
        5\})$  
      \end{enumerate}
    \end{csolution}
  \end{cexercise}
\end{csection}

\begin{csection}[id=product-sets]{Product sets}
\begin{cpara}
  Given two sets $A$ and $B$ their {\term{product}} is the set of ordered pairs $\Cvector{a,b}$,
  i.e.
  \[\Ceq{\Ccartesianproduct{A,B},\CsetCond{a,b}{\Cin{a}{A}\;\mbox{and}\;\Cin{b}{B}}{\Cvector{a,b}}}\]
  As an example, if $R$ is the real numbers, $\Ccartesianproduct{R,R}$ is the set of
  ordered pairs of numbers, $\Cvector{x,y}$ that can be used to represent the two-dimensional
  plane. Similarly, one can form the product of three copies of $R$ as
  $\Ccartesianproduct{R,R,R}$ to obtain triples of numbers that can be used to represent
  three-dimensional space. This idea of representing geometric space by pairs or triples
  of numbers is due to Descartes, and the product is often called the {\term{Cartesian
      product}}.
\end{cpara}
\begin{cpara}
   Although the space we live in seems to have three dimensions, there's nothing to stop us
   forming a product such as $\Ccartesianproduct{R,R,R,R}$ to obtain a 4-dimensional
   space. This could very useful for, say, an economic systems where there are $n$
   variables. These could be represented by an n-tuple, $\Cvector{\Cselector{x}{1},
   \Cselector{x}{2}, \dots, \Cselector{x}{n}}$ as points in an n-dimensional space.
 \end{cpara}
 \begin{cpara}
   It is not necessary for all the sets to be the same in a product.
 \end{cpara}
 \begin{cpara}
   Given a class of sets $\Ceq{C,\CsetRes{\Cselector{C}{i}}{\Cin{i}{\Cset{1, 2,
           3, \dots, n}}}}$, their {\term{product}} is the set of n-tuples 
   $\CsetCond{\Cvector{\Cselector{c}{1},\Cselector{c}{2},\dots,\Cselector{c}{n}}}
             {\Cin{\Cselector{c}{1}}{\Cselector{C}{1}}, 
              \Cin{\Cselector{c}{2}}{\Cselector{C}{2}},\dots, 
              \Cin{\Cselector{c}{n}}{\Cselector{C}{1}}}
             {\Cvector{\Cselector{c}{1},\Cselector{c}{2},\dots,\Cselector{c}{n}}}$
 \end{cpara}
 \begin{cpara}%\Pi_{\Cin{i}{I_{C}}}
   The product of an indexed class of sets, $C$, is written
   $\Cselector{\Pi}{\Cin{i}{\Cselector{I}{C}}} \Cselector{C}{i}$. 
 \end{cpara}
 \begin{cexercise}[id=saq9,name=SAQ]
   \begin{cproblem}[id=saq9p]
     \begin{enumerate}
     \item What the product of the sets $\Cset{a, b, c}$ and $\Cset{\Ccn{1},\Ccn{2}}$?
     \item What is the product of the set
       $\Cset{\Cselector{a}{\Ccn{1}},\Cselector{a}{\Ccn{2}},\Cselector{a}{\Ccn{3}}}$ and
       $\Cset{\Cselector{b}{\Ccn{1}},\Cselector{b}{\Ccn{2}}}$?  
     \item What is the product set of 
       $\CsetRes{\Cselector{a}{i}}{\Cin{i}{\mathbf{I}}}$ and
       $\CsetRes{\Cselector{b}{j}}{\Cin{j}{\mathbf{J}}}$?
     \end{enumerate}
   \end{cproblem}
   \begin{csolution}[id=saq9s]
     \begin{enumerate}
     \item What the product of the sets $\Cset{a, b, c}$ and $\Cset{\Ccn{1},\Ccn{2}}$?
       $\Ceq{\Ccartesianproduct{\Cset{a, b, c},\Cset{\Ccn{1},\Ccn{2}}},\Cset{\Cvector{a, \Ccn{1}},
           \Cvector{a, \Ccn{2}}, \Cvector{b, \Ccn{1}}, \Cvector{b, \Ccn{2}}, \Cvector{c, \Ccn{1}}, \Cvector{c, \Ccn{2}}}}$ 
     \item What is the product of the set
       $\Cset{\Cselector{a}{\Ccn{1}},\Cselector{a}{\Ccn{2}},\Cselector{a}{\Ccn{3}}}$ and
       $\Cset{\Cselector{b}{\Ccn{1}},\Cselector{b}{\Ccn{2}}}$?
       $\Ceq{\Ccartesianproduct{\Cset{\Cselector{a}{\Ccn{1}},\Cselector{a}{\Ccn{2}},
                                      \Cselector{a}{\Ccn{3}}},
                                \Cset{\Cselector{b}{\Ccn{1}},\Cselector{b}{\Ccn{2}}}},
             \Cset{\Cvector{\Cselector{a}{\Ccn{1}}, \Cselector{b}{\Ccn{1}}},
               \Cvector{\Cselector{a}{\Ccn{1}}, \Cselector{b}{\Ccn{2}}}, \Cvector{\Cselector{a}{\Ccn{2}},
               \Cselector{b}{\Ccn{1}}}, \Cvector{\Cselector{a}{\Ccn{2}}, \Cselector{b}{\Ccn{2}}},
               \Cvector{\Cselector{a}{\Ccn{3}}, \Cselector{b}{\Ccn{1}}}, \Cvector{\Cselector{a}{\Ccn{3}},
               \Cselector{b}{\Ccn{2}}}}}$ 
%        $\Ceq{\Ccartesianproduct{\Cset{\Cselector{a}{\Ccn{1}},\Cselector{a}{\Ccn{2}},
%                                       \Cselector{a}{\Ccn{3}}},
%                                 \Cset{\Cselector{b}{\Ccn{1}},\Cselector{b}{\Ccn{2}}}},
%              \Cset{\Cvector{\Cselector{a}{\Ccn{1}}, \Cselector{b}{\Ccn{1}}},
%                \Cvector{\Cselector{a}{\Ccn{1}}, \Cselector{b}{\Ccn{2}}}, \Cvector{\Cselector{a}{\Ccn{2}},
%                \Cselector{b}{\Ccn{1}}}, \Cvector{\Cselector{a}{\Ccn{2}}, \Cselector{b}{\Ccn{2}}},
%                \Cvector{\Cselector{a}{\Ccn{3}}, \Cselector{b}{\Ccn{1}}}, \Cvector{\Cselector{a}{\Ccn{3}},
%                \Cselector{b}{\Ccn{2}}}}$ 
     \item What is the product of the set $\CsetRes{\Cselector{a}{i}}{\Cin{i}{\mathbf{I}}}$ and
       $\CsetRes{\Cselector{b}{j}}{\Cin{j}{\mathbf{J}}}$? 
       $\Ceq{\Ccartesianproduct{\CsetRes{\Cselector{a}{i}}{\Cin{i}{\mathbf{I}}}, 
                                \CsetRes{\Cselector{b}{j}}{\Cin{j}{\mathbf{J}}},  
                                \CsetCond{\Cselector{a}{i}, \Cselector{b}{j}}
                                         {\;\mbox{for all}\; \Cin{i}{\mathbf{I}}\; \mbox{and all}\; 
                                               \Cin{j}{\mathbf{J}}}
                                         {\Cvector{\Cselector{a}{i}, \Cselector{b}{j}}}}}$
     \end{enumerate}
   \end{csolution}
 \end{cexercise}
\end{csection}

\begin{csection}[id=theorems-chains-reasoning]{Theorems and chains of reasoning}
  \begin{cpara}
    Generally in mathematics, things can be deduced from definitions, and we call such
    deductions {\term{theorems}}. For example
  \begin{crule}[id=theo1,type=Theorem]
    \begin{statement}[id=theo1s]
      \begin{enumerate}
      \item for all sets, $A$, $\Csubset{A,A}$
      \item for all sets $A$, $B$ and $C$, if $\Csubset{A,B}$ and $\Csubset{B,C}$ then
        $\Csubset{A,C}$ 
      \item $\Csubset{A,B}$ and $\Csubset{B,A}$ if and only\footnote{``if and only if'' is used
          so often in mathematics that it has its own abbreviation as iff. It means the
          implication goes both ways. Here it can be read as if $\Ceq{A,B}$ then $\Csubset{A,B}$ and
          $\Csubset{B,A}$, and if $\Csubset{A,B}$ and $\Csubset{B,A}$ if then $\Ceq{A,B}$.}
        if $\Ceq{A,B}$.
      \end{enumerate}
    \end{statement}
    \begin{proof}[id=theo1p]
      \begin{cpara}
        \begin{enumerate}
        \item By definition, $\Csupset{A,B}$ if and only if $\Cin{x}{A}$ implies
          $\Cin{x}{B}$. But $\Cin{x}{A}$ implies $\Cin{x}{A}$, so by definition
          $\Csubset{A,A}$. 
        \item Suppose $\Csubset{A,B}$ and $\Csubset{B,C}$. Then $\Cin{a}{A}$ implies
          $\Cin{a}{B}$. But $\Cin{a}{B}$ implies $\Cin{a}{C}$. Therefore then $\Cin{a}{A}$
          implies $\Cin{a}{C}$ and $\Csubset{A,C}$.
        \item
          \begin{itemize}
          \item{Part 1:} $\Csubset{A,B}$ and $\Csubset{B,A}$ implies $\Ceq{A,B}$. Suppose
            $\Csubset{A,B}$ and $\Csubset{B,A}$. Then $\Cin{a}{A}$ implies $\Cin{a}{B}$ and
            every element in $A$ is an element in $B$. Similarly, $\Cin{b}{B}$ implies
            $\Cin{b}{A}$ and every element in $B$ belongs to $A$. Thus $\Ceq{A,B}$.
          \item{Part 2:} $\Ceq{A,B}$ implies $\Csubset{A,B}$ and $\Csubset{B,A}$. Suppose
            $\Ceq{A,B}$. Then, by (1), $\Csubset{A,B}$, If $\Ceq{A,B}$ then $\Ceq{B,A}$,
            Since $\Ceq{B,A}$, again by (1), $\Csubset{B,A}$.
          \end{itemize}
        \end{enumerate}
      \end{cpara}
    \end{proof}
    \end{crule}
  \end{cpara}

  \begin{cpara}
    Things like $\Csubset{A,A}$ in (1) seem so self-evident sometimes it's hard to
    construct a proof. The proof here involves showing that all sets satisfy the
    requirements of the definition.
  \end{cpara}
  \begin{cpara}
    The proof of (2) depends on the transitivity of `implies'. In logic, if $p$ implies
    $q$ and $q$ implies $r$ then $p$ implies $r$. Here $p$ is ``$\Cin{a}{A}$'', $q$ is
    ``$\Cin{a}{B}$'', and $r$ is ``$\Cin{a}{C}$''. So we deduce if $\Cin{a}{A}$ then
    $\Cin{a}{C}$ which is the condition in the definition for $\Csubset{A,C}$.
  \end{cpara}
  \begin{cpara}
    Part 1 of (3) underlies a standard technique to show two sets are equal: first one
    shows that $A$ is a subset of $B$ and then one shows that $B$ is a subset of $A$. 
  \end{cpara}
  \begin{cpara}
    Part 2 illustrates a typical sleight of hand in mathematical proofs. Where did ``If
    $\Ceq{A,B}$ then $\Ceq{B,A}$'' come from? 
    It's `obvious', but it has not been proved. Can you prove it?
  \end{cpara}
  \begin{cpara}
    Finally, as usual, the little back square above is used to make it clear that the
    formal part of the theorem and its proof are finished.
  \end{cpara}

  \begin{cexercise}[id=saq10,name=SAQ]
    \begin{cproblem}[id=saq10p]
      \begin{enumerate}
      \item Show that for all sets $A$, $B$ and $C$, if $\Csubset{A,B}$ and $\Csubset{B,C}$
        then $\Csubset{A,C}$ 
      \end{enumerate}
    \end{cproblem}
    \begin{csolution}[id=saq10s]
      Show that for all sets $A$, $B$ and $C$, if $\Csubset{A,B}$ and $\Csubset{B,C}$ then
      $\Csubset{A,C}$. Let $\Cin{x}{A}$. Then $\Cin{x}{B}$ because
      $\Csubset{A,B}$. $\Cin{x}{B}$ requires $\Cin{x}{C}$ because $\Csubset{B,C}$. Thus
      $\Csubset{A,C}$. Now we have too prove that $\Cneq{A}{B}$ to obtain
      $\Csubset{A,C}$. Since $\Csubset{A,B}$ there is an element $\Cin{b}{B}$ and
      $\Cnotin{b}{A}$. Since $\Csubset{B,C}$ $\Cin{b}{C}$. Thus there exists $\Cin{b}{C}$
      and $\Cnotin{b}{A}$ , so that $\Cneq{A}{C}$, and $A$ is a proper subset of $C$,
      $\Csubset{A,C}$.
    \end{csolution}
  \end{cexercise}

  \begin{cpara}
    Set theory has two beautiful equalities called {\term{De Morgan's Laws}} after their
    discoverer, Augustus DeMorgan (1806-1871):
  \end{cpara}
  \begin{cpara}
    \begin{crule}[id=theo2,type=Theorem]
      \begin{statement}[id=theo2s]
        \begin{cpara}
        \begin{eqnarray*}
          &&\Ceq{\Ccomplement{\Cintersect{A,B}},\Cunion{\Ccomplement{A},\Ccomplement{B}}} \\
          &&\Ceq{\Ccomplement{\Cunion{A,B}},\Cintersect{\Ccomplement{A},\Ccomplement{B}}}
        \end{eqnarray*}
      \end{cpara}
    \end{statement}
    \begin{cexample}[id=theo2e]
      \begin{cpara}
        The first of these is illustrated in Figure {\ref{deMorgan}} where the complements
        are shown shaded. The union of the two complements on the right is anything that
        is dark in either $\Ccomplement{A}$ or $\Ccomplement{B}$, and this is the same as
        the complement of the intersection of $A$ and $B$.
      \end{cpara}
      \begin{cpara}
        Sometimes when I look at this diagram, it's obvious that
        $\Ceq{\Ccomplement{\Cintersect{A,B}},\Cunion{\Ccomplement{A},\Ccomplement{B}}}$,
        but other times I just can't see it (although I've known De Morgan's Laws for
        years). So, don't worry if the diagram is not very illuminating for you.
        \cfigure[id=deMorgan,%
        type=image/pdf,%
        caption={Illustration of De Morgan's Law}]%
        {width=.9\textwidth}{img/deMorgan}%
      \end{cpara}
    \end{cexample}
    \begin{proof}[id=theo2p]
      \begin{cpara}
        But is De Morgan's Law really true? Is it true for all sets, and not just these
        pictures? Let's try to show that it is.
      \end{cpara}
      \begin{cpara}
        Let $A$ and $B$ be any sets in a universe $\mathbf{U}$.
        \begin{enumerate}
        \litem{step1} By definition $\Ceq{\Cintersect{A,B},\CsetRes{x}{\Cin{x}{A}\; \mbox{and}\;
              \Cin{x}{B}}}$,
        \litem{step2} (\ref{step1}) implies that $\Cnotin{x}{\Cintersect{A,B}}$ if and only if
          $\Cnotin{x}{A}$ or $\Cnotin{x}{B}$ or both.
        \litem{step3} By definition of complements, $\Cnotin{x}{A}$ or $\Cnotin{x}{B}$ in
          (\ref{step2}) if and only if $\Cin{x}{\Ccomplement{A}}$ or
          $\Cin{x}{\Ccomplement{B}}$.
        \litem{step4} (\ref{step2}) and (\ref{step3}) imply that $\Cnotin{x}{\Cintersect{A,B}}$ if
          and only if $\Cin{x}{\Ccomplement{A}}$ or
          $\Cin{x}{\Ccomplement{B}}$.
        \litem{step5} By definition of complements,
        $\Ceq{\Ccomplement{\Cintersect{A,B}},
              \CsetRes{x}{\Cin{x}{\mathbf{U}}\;\mbox{and}\;\Cnotin{x}{\Cintersect{A,B}}}}$.
        \litem{step7} Substituting (\ref{step4}) into (\ref{step5}) gives
          $\Ceq{\Ccomplement{\Cintersect{A,B}}, 
                \CsetRes{x}{\Cin{x}{\mathbf{U}}\;\mbox{and}\;
                                      \Cvector{\Cin{x}{\Ccomplement{A}\;\mbox{or}\;\Cin{x}{\Ccomplement{B}}}}}}$
        \litem{step8} But $\CsetRes{x}{\Cin{x}{\mathbf{U}}\; \mbox{and}\;
            (\Cin{x}{\Ccomplement{A}}\; \mbox{or}\; \Cin{x}{\Ccomplement{B}})}$ is the
          set $\Cunion{\Ccomplement{A},\Ccomplement{B}}$. So
        \litem{step9} 
        $\Ceq{%\Ccomplement{\Cintersect{A,B}}, 
              \CsetRes{x}{\Cin{x}{\mathbf{U}}\; \mbox{and}\; 
                      (\Cin{x}{\Ccomplement{A}}\;\mbox{or}\;\Cin{x}{\Ccomplement{B}})},
              \Cunion{\Ccomplement{A},\Ccomplement{B}}}$
        \litem{step10} So,
          $\Ceq{\Ccomplement{\Cintersect{A,B}},\Cunion{\Ccomplement{A},\Ccomplement{B}}}$
        \end{enumerate}
      \end{cpara}
    \end{proof}
  \end{crule}
\end{cpara}
  
\begin{cexercise}[id=saq11,name=SAQ]
     \begin{cproblem}[id=saq11p]
       \begin{enumerate}
       \item Show $\Ceq{\Ccomplement{\Cintersect{A,B,C}},\Cunion{\Ccomplement{A},\Ccomplement{B},\Ccomplement{C}}}$ ?
       \end{enumerate}
     \end{cproblem}
     \begin{csolution}[id=saq11s]
       Show
       $\Ceq{\Ccomplement{\Cintersect{A,B,C}},\Cunion{\Ccomplement{A},\Ccomplement{B},\Ccomplement{C}}}$?
       Let $\Ceq{D,\Cintersect{B,C}}$. Then by de Morgan's laws,
       $\Ceq{\Ccomplement{\Cintersect{A,D}},\Cunion{\Ccomplement{A},\Ccomplement{D}}}$. $\Ceq{\Cunion{\Ccomplement{A},
           \Ccomplement{D}}, \Cunion{\Ccomplement{A},\Ccomplement{\Cintersect{B,C}}}}$. By de Morgan's
       Laws $\Ceq{\Cunion{\Ccomplement{A},\Ccomplement{\Cintersect{B,C}}},
         \Cunion{\Ccomplement{A},\Cintersect{\Ccomplement{B},\Ccomplement{C}}},
         \Cunion{\Ccomplement{A},\Ccomplement{B},\Ccomplement{C}}}$
     \end{csolution}
   \end{cexercise}
\end{csection}

\begin{csection}[id=multiple-choice-questions]{Multiple choice questions}
\begin{cpara}
  You have now completed all the formal material for this session. Answer each of the
  following multiple choice questions, by selecting what you think is the right answer. If
  you cannot answer the question circle the x.
\end{cpara}
\begin{cpara}
    Each multiple choice question has one answer. Tick your choices
  \end{cpara}
  
  \begin{cexercise}[id=msq1,name=Q]
    \begin{cproblem}[id=msq1p]
      the set $\Cset{a, e, i, o, u}$ is defined
      \begin{enumerate}
      \item explicitly
      \item extensionally
      \item by the listing convention
      \item intensionally
      \item independently
      \item internally
      \item elementally
      \item externally
      \item[(x)] don't know
      \end{enumerate}
    \end{cproblem}
  \end{cexercise}
  \begin{cexercise}[id=msq2,name=Q]
    \begin{cproblem}[id=msq2p]
      Who is said to have invented the equals sign?
      \begin{enumerate}
      \item Pascal
      \item Fermat
      \item Leonardo
      \item Vi\`ete
      \item Boyer
      \item Recorde
      \item Russ
      \item Newton
      \item[(x)] don't know
      \end{enumerate}
    \end{cproblem}
  \end{cexercise}
  \begin{cexercise}[id=msq3,name=Q]
    \begin{cproblem}[id=msqp3p]
      What is the correct way to read $\Cin{x}{\Csubset{\Cunion{A,B},\Csetdiff{C}{D}}}$
      \begin{enumerate}
      \item $x$ belongs to the intersection of $A$ and $B$ which is a subset of $C$ minus $D$
      \item $x$ belongs to the union of $A$ and $B$ which is a proper subset of $D$ minus $C$
      \item $x$ is a subset of $\Cunion{A,B}$ which belongs to the symmetric difference of $C$ and $D$
      \item $x$ doesn't belong to the union of $A$ and $B$ which is a subset of the difference of $C$ minus $D$
      \item $x$ belongs to the intersection of $A$ and $B$ which is a subset of the difference of $C$ minus $D$
      \item $x$ belongs to the union of $A$ and $B$ which is a subset of the difference of set $C$ minus set $D$
      \item $x$ and $C$ minus $D$ belong to the union of $A$ and $B$
      \item The symmetric difference of $C$ and $D$ contains $A$ union $B$ which contains $x$ 
      \item[(x)] don't know
      \end{enumerate}
    \end{cproblem}
  \end{cexercise}
  \begin{cexercise}[id=msq4,name=Q]
    \begin{cproblem}[id=msq4p]
      Given a set $X$ in a universe $\mathbf{U}$, the complement of $X$ is
      \begin{enumerate}
      \item $\Ceq{\Ccomplement{X}, \Csetdiff{X}{\mathbf{U}}}$
      \item $\Ceq{\Ccomplement{X}, \Csetdiff{\mathbf{U}}{X}}$
      \item $\Ceq{\Ccomplement{X}, \Csetdiff{\Ccomplement{\mathbf{U}}}{X}}$
      \item $\Ceq{\Ccomplement{X}, \Cunion{X,\mathbf{U}}}$
      \item $\Ceq{\Ccomplement{X}, \Cintersect{X,\mathbf{U}}}$
      \item $\Ceq{\Ccomplement{X}, \Csubset{X,\mathbf{U}}}$
      \item[(x)] don't know
      \end{enumerate}
    \end{cproblem}
  \end{cexercise}
  \begin{cexercise}[id=msq5,name=Q]
    \begin{cproblem}[id=msq5p]
      Which of the following is correct
      \begin{enumerate}
      \item $\Ceq{A \Delta B, \Cunion{\Ccomplement{A},\Ccomplement{B}}}$
      \item $\Ceq{A \Delta B, \Cintersect{\Ccomplement{A},\Ccomplement{B}}}$
      \item $\Ceq{A \Delta B, \Cunion{\Csetdiff{A}{B},\Csetdiff{B}{A}}}$
      \item $\Ceq{A \Delta B, \Cintersect{\Csetdiff{A}{B},\Csetdiff{B}{A}}}$
      \item $\Ceq{A \Delta B, \Csetdiff{\Csetdiff{A}{B}}{\Csetdiff{B}{A}}}$
      \item $\Ceq{A \Delta B, \Csetdiff{A}{B} + \Csetdiff{B}{A}}$ % where has + been defined?
      \item[(x)] don't know
      \end{enumerate}
    \end{cproblem}
  \end{cexercise}

  \begin{cexercise}[id=msq6,name=Q]
    \begin{cproblem}[id=msq6p]
      The power set of the set $\Cset{a, b, c}$ is
      \begin{enumerate}
      \item $\Cemptyset$
      \item $\Cset{ \Cset{a}, \Cset{b}, \Cset{c}}$
      \item $\Cset{ \Cset{a, b}, \Cset{a, c}, \Cset{b, c}}$
      \item $\Cset{ \Cset{a}, \Cset{b}, \Cset{c}, \Cset{a, b}, \Cset{a, c}, \Cset{b, c}}$
      \item $\Cset{ \Cset{a}, \Cset{b}, \Cset{c}, \Cset{a, b}, \Cset{a, c}, \Cset{b, c}, \Cset{a, b, c}}$
      \item $\Cset{ \Cemptyset, \Cset{a}, \Cset{b}, \Cset{c}, \Cset{a, b}, \Cset{a, c}, \Cset{b, c}, \Cset{a, b, c}}$
      \item $\Cpower{2}{\Cset{ \Cset{a}, \Cset{b}, \Cset{c}}}$
      \item $\Capply{P}X$ 
      \item[(x)] don't know
      \end{enumerate}
    \end{cproblem}
  \end{cexercise}

  \begin{cexercise}[id=msq7,name=Q]
    \begin{cproblem}[id=msq7p]
      The product of the sets of $\Cset{a, b, c}$ and $\Cset{x, y, z}$ is
      \begin{enumerate}
      \item $\Cset{\Cvector{a, x}, \Cvector{a, y}, \Cvector{a, z}, \Cvector{b, x}, \Cvector{b, y}, \Cvector{b, z}, \Cvector{c, x}, \Cvector{c, y}, \Cvector{c, z}}$
      \item $\Cset{\Cvector{a, x}, \Cvector{b, y}, \Cvector{c, z}}$
      \item $\Cset{\Cvector{a, b, c}, \Cvector{x, y, z}}$
      \item $\Cset{\Cvector{a, b}, \Cvector{a, c}, \Cvector{b, c}, \Cvector{x, y}, \Cvector{x, z}, \Cvector{y, z}}$
      \item $\Cset{ \Cvector{\Ctimes{a,x}}, \Cvector{\Ctimes{a,y}}, \Cvector{\Ctimes{a,z}}, \Cvector{\Ctimes{b,x}},
          \Cvector{\Ctimes{b,y}}, \Cvector{\Ctimes{b,z}}, \Cvector{\Ctimes{c,x}}, \Cvector{\Ctimes{c,y}}, \Cvector{\Ctimes{c,z}}}$
      \item[(x)] don't know
      \end{enumerate}
    \end{cproblem}
  \end{cexercise}
  \begin{cexercise}[id=msq8,name=Q]
    \begin{cproblem}[id=msq8p]
      Which of the following is true
      \begin{enumerate}
      \item if $\Csubset{A,B}$ and $\Csubset{B,C}$ then $\Ceq{A,C}$
      \item if $\Csubset{A,B}$ and $\Csubset{B,C}$ then $\Csubset{A,C}$
      \item if $\Ceq{A,B}$ and $\Csubset{B,C}$ then $\Ceq{A,C}$
      \item if $\Csubset{A,B}$ and $\Csubset{B,C}$ then $\Csubset{A,C}$
      \item if $\Cnotsubset{A}{B}$ and $\Cnotsubset{B}{C}$ then $\Cnotsubset{A}{C}$
      \item[(x)] don't know
      \end{enumerate}
    \end{cproblem}
  \end{cexercise}
  \begin{cexercise}[id=msq9,name=Q]
    \begin{cproblem}[id=msq9p]
      Which of the following is correct?
      \begin{enumerate}
      \item $\Ceq{\Ccomplement{\Cintersect{A,B}},\Cunion{A,B}}$
      \item $\Ceq{\Ccomplement{\Cintersect{A,B}},\Cintersect{\Ccomplement{A},\Ccomplement{B}}}$
      \item $\Ceq{\Ccomplement{\Cintersect{A,B}},\Ccomplement{A} \Delta \Ccomplement{B}}$
      \item $\Ceq{\Ccomplement{\Cintersect{A,B}},\Csetdiff{\Ccomplement{A}}{\Ccomplement{B}}}$
      \item $\Ceq{\Ccomplement{\Cintersect{A,B}},\Cunion{\Ccomplement{A},\Ccomplement{B}}}$
      \item $\Ceq{\Ccomplement{\Cintersect{A,B}},\Cintersect{A,B}}$
      \item[(x)] don't know
      \end{enumerate}
    \end{cproblem}
  \end{cexercise}
  \begin{cexercise}[id=msq10,name=Q]
    \begin{cproblem}[id=msq10p]
      Which of the following is correct?
      \begin{enumerate}
      \item $\Ceq{\Ccomplement{\Cunion{A,B}}, A \cap B}$
      \item $\Ceq{\Ccomplement{\Cunion{A,B}}, \Ccomplement{A} + \Ccomplement{B}}$
      \item $\Ceq{\Ccomplement{\Cunion{A,B}}, \Cunion{\Ccomplement{A},\Ccomplement{B}}}$
      \item $\Ceq{\Ccomplement{\Cunion{A,B}}, \Cunion{\Ccomplement{A},\Ccomplement{B}}}$
      \item $\Ceq{\Ccomplement{\Cunion{A,B}}, \Csetdiff{\Ccomplement{A}}{\Ccomplement{B}}}$
      \item $\Ceq{\Ccomplement{\Cunion{A,B}}, \Cunion{A,B}}$
      \item[(x)] don't know
      \end{enumerate}
    \end{cproblem}
  \end{cexercise}
\end{csection}

\begin{csection}[id=conclusion-session-1]{Conclusion to Session 1}
\begin{cpara}
  This concludes your work in Session 1. During it you have seen the following
  \begin{itemize}
  \item the definitions of elements, sets and classes
  \item the set membership and subset relationships
  \item the intersection and union relationships
  \item the complement of a set in its universe
  \item the use of Venn diagrams
  \item the power set of a set
  \item the product of two or more sets
  \end{itemize}
  This is a lot of material to cover in one session, and it has generated a lot of new
  notation. Although all of this may be new to you, I hope you have grasped some of the
  basic ideas, and that you feel you can `read' the notation and make sense of it. 
\end{cpara}
\begin{cpara}
    To finish this Session, please go through your answers to the multiple-choice questions,
    and enter them on the End of Session survey along with your other answers and email the
    survey to 
  \begin{center}
    \link[src=http://complexity@open.ac.uk]{http://complexity@open.ac.uk}
  \end{center}
  \end{cpara}
\end{csection}
\end{ccontent}
\end{cnxmodule}
\end{document}
